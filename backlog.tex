\documentclass[12pt]{article}

\usepackage{amsmath,amssymb,amsfonts} %In case we need math stuff
\usepackage{graphicx} %For inserting images and stuff
\usepackage{listings} %For inserting code snippets
\usepackage{enumerate}
\usepackage[margin=2cm]{geometry} %Nice margin setting

\title{ECSE 321 - Intro to Software Engineering\\Backlog Week 1}
\author{Harley Wiltzer\\Camilo Garcia La Rotta\\Jacob Shnaidman\\Robert Attard\\Matthew Lesko}
\date{February 17, 2017}

\begin{document}
\pagenumbering{gobble} %No page number on title page
\maketitle
\newpage
\pagenumbering{arabic} %Arabic numeral page numbers on regular pages

\section{Week 1}

\subsection{Main Tasks}

%\textbf{ADD HERE ALL THE SHIT YALL DID}

\begin{itemize}
    \item Harley Wiltzer
	\begin{itemize}
		\item Co-formulated the prototype Requirements Document.
		\item Drafted the first Requirements Document.
		\item Designed Sequence Diagrams for Instructor Actions, Student Actions, Admin Actions.
		\item Made small changes to Class Diagram.
	\end{itemize}
    \item Camilo Garcia La Rotta
    \begin{itemize}
        \item Co-formulated the prototype Requirements Document
        \item Designed the prototype Class Diagram
        \item Structured the GitHub file system
    \end{itemize}
    \item Jacob Shnaidman
		\begin{itemize}
			\item Designed Use Case Diagrams
			\item Wrote Use Case Descriptions
			\item Helped with the design of Sequence Diagrams
			\item Wrote the Work Plan
		\end{itemize}
    \item Robert Attard
    \begin{itemize}
        \item Assisted in preliminary design of the use case diagrams
        \item Designed and improved on the state diagram
        \item Made wording and grammatical changes to Requirements Document
    \end{itemize}
    \item Matthew Lesko
    \begin{itemize}
    	\item Designed the prototype Use Case Diagram
    	\item Co-designed Sequence Diagrams
    \end{itemize}
\end{itemize}

\subsection{Key Decisions}

%\textbf{ADD HERE ALL THE DECISIONS YALL TOOK}

\begin{enumerate}
    \item \textbf{Requirement Document}
    \item \textbf{Use Cases}
    \begin{itemize}
    	\item Included four actors in the diagram, three of which are users and one of which is a service. The three users are: Student, Instructor and Department (has administrator privileges). The service is a Security Authentication.
    	\item Each user in the diagram acts upon multiple use cases respective to their functionalities mentioned in the Requirements and Specifications documents. Each use case describes functionality that its respective actor has.
    \end{itemize}
    \item \textbf{Class Diagram}
    \begin{itemize}
        \item Trying to mimic the architecture of the event registration app, an early choice was to have a JobManager oversee the job transactions between Instructors and Students
        \item Each user, whether it has administrator, instructor or student access rights, is obliged to have profile instance to facilitate identity access management
		\item Later, it was also decided to have a ProfileManager to handle the actors' profiles and
			their authentication.
        \item To enable an administrator to recreate actions available to instructors or student, we give the class administrator the power to create instances of the aforementioned users
    \end{itemize}
    \item \textbf{Sequence Diagram}
		\begin{itemize}
			\item For convenience and concision, it was decided to design sequence diagrams for each
				main user action.
			\item The sequence diagrams may contain actions from multiple user classes to emphasize
				the sequence. For example, in the Instructor sequence diagram, the action of an
				administrator verifying Instructor modifications to the TA/Grader hours is seen.
			\item The current sequence diagrams include Student actions, Instructor actions,
				Administrator actions, and the authentication process.
		\end{itemize}
    \item \textbf{State Chart}
        \begin{itemize}
            \item The state chart tracks the possible states of the job posting between being created and the position being filled.
            \item The chart also indicates the triggers required to transition between job posting states.

        \end{itemize}
\end{enumerate}

\subsection{Work Plan}

In the following week, we will create more detailed sequence diagrams going into the details of the implementations on individual platforms. We will then develop prototype source code for the Publish Job Posting and Apply for Job use cases for the desktop and mobile applications respectively.

\subsection{Work Hours}

\begin{itemize}
    \item Harley Wiltzer
		\begin{itemize}
			\item Requirements document: 1 hours
			\item Sequence diagrams: 3 hours
			\item Class diagram: 30 minutes
			\item Compilation of report: 4 hours
		\end{itemize}
    \item Camilo Garcia La Rotta
    \begin{itemize}
        \item Requirements document: 2 hours
        \item Class diagram: 2 hours
        \item State diagram: 1 hour
    \end{itemize}
    \item Jacob Shnaidman
    \begin{itemize}
        \item Use Case Diagrams: 4 hours
        \item Use Case Description 1.5 hours
		\item Helped with Sequence Diagrams 0.5 hours
		\item Work Plan 0.5 hours
    \end{itemize}
    \item Robert Attard
    \begin{itemize}
        \item Requirements document: 0.5 hours
        \item Use Case diagram: 3.5 hours
        \item State diagram: 3 hours
    \end{itemize}
    \item Matthew Lesko
    \begin{itemize}
    	\item Use case diagram: 4 hours
    	\item Co-design sequence diagrams: 4 hours
    \end{itemize}
\end{itemize}

\end{document}
