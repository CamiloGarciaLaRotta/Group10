\documentclass[12pt]{article}

\usepackage{amsmath,amssymb,amsfonts} %In case we need math stuff
\usepackage{graphicx} %For inserting images and stuff
\usepackage{listings} %For inserting code snippets
\usepackage{enumerate}
\usepackage[margin=2cm]{geometry} %Nice margin setting

\title{ECSE 321 - Intro to Software Engineering\\Backlog Week 1}
\author{Harley Wiltzer\\Camilo Garcia La Rotta\\Jacob Shnaidman\\Robert Attard\\Matthew Lesko}
\date{February 17, 2017}

\begin{document}
\pagenumbering{gobble} %No page number on title page
\maketitle
\newpage
\pagenumbering{arabic} %Arabic numeral page numbers on regular pages

\section{Deliverable 1}

\subsection{Main Tasks}

%\textbf{ADD HERE ALL THE SHIT YALL DID}

\begin{itemize}
    \item Harley Wiltzer
	\begin{itemize}
		\item Co-formulated the prototype Requirements Document.
		\item Drafted the first Requirements Document.
		\item Designed Sequence Diagrams for Instructor Actions, Student Actions, Admin Actions.
		\item Made small changes to Class Diagram.
	\end{itemize}
    \item Camilo Garcia La Rotta
    \begin{itemize}
        \item Co-formulated the prototype Requirements Document
        \item Designed the prototype Class Diagram
        \item Structured the GitHub file system
    \end{itemize}
    \item Jacob Shnaidman
		\begin{itemize}
			\item Designed Use Case Diagrams
			\item Wrote Use Case Descriptions
			\item Helped with the design of Sequence Diagrams
			\item Wrote the Work Plan
		\end{itemize}
    \item Robert Attard
    \begin{itemize}
        \item Assisted in preliminary design of the use case diagrams
        \item Designed and improved on the state diagram
        \item Made wording and grammatical changes to Requirements Document
    \end{itemize}
    \item Matthew Lesko
    \begin{itemize}
    	\item Designed the prototype Use Case Diagram
    	\item Co-designed Sequence Diagrams
    \end{itemize}
\end{itemize}

\subsection{Key Decisions}

%\textbf{ADD HERE ALL THE DECISIONS YALL TOOK}

\begin{enumerate}
    \item \textbf{Requirement Document}
    \item \textbf{Use Cases}
    \begin{itemize}
    	\item Included four actors in the diagram, three of which are users and one of which is a service. The three users are: Student, Instructor and Department (has administrator privileges). The service is a Security Authentication.
    	\item Each user in the diagram acts upon multiple use cases respective to their functionalities mentioned in the Requirements and Specifications documents. Each use case describes functionality that its respective actor has.
    \end{itemize}
    \item \textbf{Class Diagram}
    \begin{itemize}
        \item Trying to mimic the architecture of the event registration app, an early choice was to have a JobManager oversee the job transactions between Instructors and Students
        \item Each user, whether it has administrator, instructor or student access rights, is obliged to have profile instance to facilitate identity access management
		\item Later, it was also decided to have a ProfileManager to handle the actors' profiles and
			their authentication.
        \item To enable an administrator to recreate actions available to instructors or student, we give the class administrator the power to create instances of the aforementioned users
    \end{itemize}
    \item \textbf{Sequence Diagram}
		\begin{itemize}
			\item For convenience and concision, it was decided to design sequence diagrams for each
				main user action.
			\item The sequence diagrams may contain actions from multiple user classes to emphasize
				the sequence. For example, in the Instructor sequence diagram, the action of an
				administrator verifying Instructor modifications to the TA/Grader hours is seen.
			\item The current sequence diagrams include Student actions, Instructor actions,
				Administrator actions, and the authentication process.
		\end{itemize}
    \item \textbf{State Chart}
        \begin{itemize}
            \item The state chart tracks the possible states of the job posting between being created and the position being filled.
            \item The chart also indicates the triggers required to transition between job posting states.

        \end{itemize}
\end{enumerate}

\subsection{Work Plan}

In the following week, we will create more detailed sequence diagrams going into the details of the implementations on individual platforms. We will then develop prototype source code for the Publish Job Posting and Apply for Job use cases for the desktop and mobile applications respectively.

\subsection{Work Hours}

\begin{itemize}
    \item Harley Wiltzer
		\begin{itemize}
			\item Requirements document: 1 hours
			\item Sequence diagrams: 3 hours
			\item Class diagram: 30 minutes
			\item Compilation of report: 4 hours
		\end{itemize}
    \item Camilo Garcia La Rotta
    \begin{itemize}
        \item Requirements document: 3 hours
        \item Class diagram: 2 hours
        \item State diagram: 2 hour
    \end{itemize}
    \item Jacob Shnaidman
    \begin{itemize}
        \item Use Case Diagrams: 4 hours
        \item Use Case Description 1.5 hours
		\item Helped with Sequence Diagrams 0.5 hours
		\item Work Plan 0.5 hours
    \end{itemize}
    \item Robert Attard
    \begin{itemize}
        \item Requirements document: 0.5 hours
        \item Use Case diagram: 3.5 hours
        \item State diagram: 3 hours
    \end{itemize}
    \item Matthew Lesko
    \begin{itemize}
    	\item Use case diagram: 4 hours
    	\item Co-design sequence diagrams: 4 hours
    \end{itemize}
\end{itemize}

\newpage

\section{Deliverable 2}

\subsection{Main Tasks}

\begin{itemize}
    \item Harley Wiltzer
	\begin{itemize}
		\item Aided in the design of the revamped domain model
		\item Designed and developed desktop application
		\item Aided in the design of the mobile application
		\item Created the Sequence diagram for the Desktop - Apply to Job Use Case
		\item Helped create the Sequence Diagram for the Desktop - Publish Job Posting Use Case
		\item Compiled the final report
	\end{itemize}
    \item Camilo Garcia La Rotta
    \begin{itemize}
        \item Designed and documented Sequence Diagrams for the Web and Desktop implementation of "Publish Job Posting"
        \item Derived the PHP code implementation of "Publish Job Posting" and "Apply to Job Posting" based on the aforementioned design-level Class Diagram
    \end{itemize}
    \item Jacob Shnaidman
		\begin{itemize}
			\item Co-designed preliminary Architecture Block Diagram
            \item Helped with design of Detailed Design Diagram
            \item Wrote the Java code in Android Studio for the implementation of "Apply to Job Posting" 
            \item Designed and documented Sequence Diagrams for the Mobile implementation of "Apply To Job Posting"
		\end{itemize}
    \item Robert Attard
    \begin{itemize}
        \item Co-designed design level class diagram
       	\item Wrote natural language descriptions for publish use-case workflow
       	\item Created view package class diagrams
    \end{itemize}
    \item Matthew Lesko
    \begin{itemize}
    	\item Co-designed preliminary Architecture Block Diagram
    	\item Wrote draft for Architecture's Description and Rationale
    	\item Co-designed Detailed Domain Model Diagram
    	\item Wrote Detailed Design report
    	\item Made Controller Package Class Diagrams
    \end{itemize}
\end{itemize}

\subsection{Key Decisions}

\begin{enumerate}
    \item \textbf{Use Cases}
    \begin{itemize}
    	\item In order to correctly implement the Entity-Control-Boundary pattern and symbolism in the Use Cases, the naming of the cases from the first deliverable were reviewed and adjusted. Moreover, the relationships between each action was refined so as to reflect the following rules:
    	\begin{itemize}
    	    \item Actor only interacts with $<<$Boundary$>>$
    	    \item $<<$Entity$>>$ only interacts with $<<$Control$>>$
    	    \item $<<$Control$>>$ manages the interaction flow between $<<$Boundary$>>$ and $<<$Entity$>>$
    	\end{itemize}
    	\item The overall structure of the Use Case for deliverable \#2 focused solely on the Profile subclasses Admin and Instructor. To highlight this structure the Use Cases made a deliberate use of Generalization.
        \item The "Publish Job Posting" Sequence diagram had to be clear about the alternative actions based on the approval or disapproval of the Posting instance. As such, an ALT pane was implemented only for the Web Use Case, which didn't necessarily imply the posting would be automatically approved. As opposed to the Desktop application, only used by Admin which ensures the posting is automatically approved at the moment of submitting a publish form.
		\item It was decided that the Desktop app can handle the creation of all major entities.
			Starting with no data, the Desktop app has the power to create all necessary actors for
			all features of the program to be possible.
		\item For the purpose of Deliverable 2, the mobile app does not have the ability to create
			student profiles. Students register for jobs by specifying their username, as
			authentication has not been implemented yet.
		\item Since the mobile app does not support the creation of Students or Jobs, it was decided
			that it should programmatically create Students and Jobs if no current persistence XML
			file is found. This was decided solely for the convenience of the graders of this
			deliverable. If this is not desired, the grader may input his/her own XML to the
			appropriate directory.
    \end{itemize}
    \item \textbf{Domain Model}
    \begin{itemize}
        \item ProfileManager, CourseManager, and ApplicationManager were added to the domain model
			to assist in the creation and persistence of Profiles, Courses, Jobs, and Applications.
    \end{itemize}
	\item \textbf{Architecture}
	\begin{itemize}
		\item The architecture was designed using two different architecture design patterns: a Model/View/Controller pattern and a Layered Architecture patter. The MVC allows one to make changes to a certain component, say the Model component, without affecting the two others. This allows for modular design. The team is also experienced with the MVC pattern, hence another reason why it was chosen.
		\item Since an authorization and authentication service is required of the system, the architecture consists of an Authorization and Authentication layer above the MVC layer. The user first interacts with the Authorization and Authentication layer, then once access is granted, the user has access to the MVC layer, which is the TAMAS registration system.
	\end{itemize}
\end{enumerate}

\subsection{Work Plan}

In the following Deliverable, thorough testing needs to be completed. The following are deadlines and the estimated effort it will take to complete these tasks on a scale of 1 to 10.

\begin{itemize}
    \item March 10 - First iteration of unit tests for Apply to Job and Publish Job posting on web, desktop and mobile. Estimated Effort: 8/10
    \item March 13 - Second iteration of unit tests for Apply to Job and Publish Job posting on web, desktop and mobile. Estimated Effort: 6/10
    \item March 13 - First iteration of System and Component tests. Estimated Effort: 6/10
    \item March 15 - First iteration of Stress Testing and Final iteration of unit tests. Estimated Effort: 6/10
    \item March 16 - First iteration of Descriptions of all tests done. Estimated Effort: 5/10
    \item March 10 - Deadline for Deliverable 3, including editing of backlog and update of workplan. Estimated Effort: 4/10

\end{itemize}

\subsection{Work Hours}

\begin{itemize}
    \item Harley Wiltzer
		\begin{itemize}
			\item Domain Model Modification: 2 hours
			\item Java Desktop App: 9 hours
			\item Android App: 2 hours
			\item Sequence Diagrams: 2 hours
			\item Compilation and Editing of Final Report: 3 hours
		\end{itemize}
    \item Camilo Garcia La Rotta
    \begin{itemize}
        \item Use Case Diagrams: 3 hours
        \item Sequence Diagrams: 2 hours
        \item PHP: 9 hours
    \end{itemize}
    \item Jacob Shnaidman
    \begin{itemize}
        \item Architecture Block Diagram: 2 hours
        \item Sequence Diagram: .66 hours
        \item Android Studio: 8 hours
    \end{itemize}
    \item Robert Attard
    \begin{itemize}
        \item Detailed Domain Model Diagram: 2 hours
    	\item View package class diagrams: 4 hours
    	\item Publish workflow description: 3 hours
    \end{itemize}
    \item Matthew Lesko
    \begin{itemize}
    	\item Architecture Block Diagram and Description and Rationale: 3 hours
    	\item Detailed Domain Model Diagram: 1 hour
    	\item Detailed Design Report and Class Diagrams: 3 hours
    \end{itemize}
\end{itemize}

 \section{Week 3}
%
 \subsection{Main Tasks}
%
 %\textbf{ADD HERE ALL THE SHIT YALL DID}
%
 \begin{itemize}
     \item Harley Wiltzer
 	\begin{itemize}
 		\item Wrote the Description section of the System Test Plan
		\item Wrote the Rationale section of the System Test Plan
		\item Wrote the Test Coverage section of the System Test Plan
		\item Wrote and designed the two detailed test cases for the System Test Plan
 	\end{itemize}
     \item Camilo Garcia La Rotta
     \begin{itemize}
        \item Wrote the Description section of the Unit Test Plan
		\item Defined the structure to present the test cases of the Unit Test Plan
		\item Wrote half the classes tested of the Unit Test Plan
		\item Wrote the Test Coverage section of the System Test Plan
		\item Wrote the Test Techniques and Tools section of the Unit Test Plan
		\item Derived next workplan
     \end{itemize}
     \item Jacob Shnaidman
 		\begin{itemize}
 			\item gg
 		\end{itemize}
     \item Robert Attard
     \begin{itemize}
         \item gg
     \end{itemize}
     \item Matthew Lesko
     \begin{itemize}
     	\item gg
     \end{itemize}
 \end{itemize}
%
 \subsection{Key Decisions}
%
 %\textbf{ADD HERE ALL THE DECISIONS YALL TOOK}
%
 \begin{enumerate}
    \item \textbf{Unit Test Plan}
    \begin{itemize}
     	\item The main decision was the design of framework of Test Cases. We aimed to have a well structured suite of test cases that could be generalized to the maximum amount of classes. This would minimize the time spent for coding the tests and facilitate the process of verification of such code. The final choice was a set of tests by-attribute rather than by-outcome
		\item Tested JaCoCo but decided to continue using EclEmma as sole Desktop code coverage tool due to its use for the course ECSE 321
		\item Tested Selenium but decided to continue using Junit as sole Desktop code coverage tool due to its use for the course ECSE 321
     \end{itemize}
     \item \textbf{System Test Plan}
     \begin{itemize}
     	\item Decided to make system tests for each use case
		\item Decided to structure system tests by actor
		\item Chose to design detailed system test plan for the Publish Job Posting and Apply To Job
			use cases, as enough infrastructure has already been programmed for a solid
			understanding of how those use cases can be tested.
		\item Derived the more general logical sequence of actions possible for system Testing based on the Requirement Document of Deliverable \#1
     \end{itemize}
     \item \textbf{Sequence Diagram}
     \begin{itemize}
         \item gg
     \end{itemize}
 \end{enumerate}
%
 \subsection{Work Plan}
%
In the following Deliverable, a formal release pipeline plan will be published. The report will contain an overview of the implementation (tools, design and rational) at each phase of the process. The following are deadlines and the estimated effort it will take to complete these tasks on a scale of 1 to 10.

\begin{itemize}
    \item March 21 - Integration phase section Starts. Estimated Effort: 8/10
    \item March 24 - Integration phase section completed. 
    \item March 25 - Build phase section starts Estimated Effort: 8/10
    \item March 28 - Build phase section completed.
    \item March 29 - Deployment section starts Estimated Effort: 8/10
    \item April 1 - Deployment phase section completed. 
    \item April 2 - Compiling final report, editing of backlog and update of workplan. Estimated Effort: 5/10
    \item April 2 - Deadline for Deliverable 4. 

\end{itemize}
%
 \subsection{Work Hours}
%
 \begin{itemize}
     \item Harley Wiltzer
 		\begin{itemize}
 			\item System Test Plan: 5 hours
 		\end{itemize}
     \item Camilo Garcia La Rotta
 		\begin{itemize}
 			\item Unit Test Plan: 5 hours
 			\item System Test Plan Coverage: 1/2 hour
 			\item Workplan: 1/2 hour
 		\end{itemize}
     \item Jacob Shnaidman
     \begin{itemize}
         \item gg
     \end{itemize}
     \item Robert Attard
     \begin{itemize}
         \item gg
     \end{itemize}
     \item Matthew Lesko
     \begin{itemize}
     	\item gg
     \end{itemize}
 \end{itemize}

%% TEMPLATE
%%% JSUT COPY PASTE THIS EVERY WEEK

% \section{Week 2}

% \subsection{Main Tasks}

% %\textbf{ADD HERE ALL THE SHIT YALL DID}

% \begin{itemize}
%     \item Harley Wiltzer
% 	\begin{itemize}
% 		\item gg
% 	\end{itemize}
%     \item Camilo Garcia La Rotta
%     \begin{itemize}
%         \item gg
%     \end{itemize}
%     \item Jacob Shnaidman
% 		\begin{itemize}
% 			\item gg
% 		\end{itemize}
%     \item Robert Attard
%     \begin{itemize}
%         \item gg
%     \end{itemize}
%     \item Matthew Lesko
%     \begin{itemize}
%     	\item gg
%     \end{itemize}
% \end{itemize}

% \subsection{Key Decisions}

% %\textbf{ADD HERE ALL THE DECISIONS YALL TOOK}

% \begin{enumerate}
%     \item \textbf{Use Cases}
%     \begin{itemize}
%     	\item gg
%     \end{itemize}
%     \item \textbf{Sequence Diagram}
%     \begin{itemize}
%         \item gg
%     \end{itemize}
% \end{enumerate}

% \subsection{Work Plan}

% gg

% \subsection{Work Hours}

% \begin{itemize}
%     \item Harley Wiltzer
% 		\begin{itemize}
% 			\item gg
% 		\end{itemize}
%     \item Camilo Garcia La Rotta
%     \begin{itemize}
%         \item gg
%     \end{itemize}
%     \item Jacob Shnaidman
%     \begin{itemize}
%         \item gg
%     \end{itemize}
%     \item Robert Attard
%     \begin{itemize}
%         \item gg
%     \end{itemize}
%     \item Matthew Lesko
%     \begin{itemize}
%     	\item gg
%     \end{itemize}
% \end{itemize}




\end{document}
