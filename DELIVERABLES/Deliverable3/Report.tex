\documentclass[12pt]{report}

\usepackage{amsmath,amssymb,amsfonts} %In case we need math stuff
\usepackage{graphicx} %For inserting images and stuff
\usepackage{listings} %For inserting code snippets
\usepackage{enumerate} %For fancy enumeration
\usepackage{hyperref}
\usepackage{pdfpages}
\usepackage{float}
\usepackage[margin=2cm]{geometry} %Nice margin setting

\renewcommand*\thesection{\arabic{section}}

\title{ECSE 321 - Intro to Software Engineering\\Design Specification Document - Deliverable 3}
\author{Harley Wiltzer\\Camilo Garcia La Rotta\\Jake Shnaidman\\Robert Attard\\Matthew Lesko}
\date{March 19, 2017}

\begin{document}
\pagenumbering{gobble} %No page number on title page
\maketitle
\newpage
\pagenumbering{arabic} %Arabic numeral page numbers on regular pages
\tableofcontents
\part{Unit Test Plan}
\section{Description}
	In this section we will present an extensive suite of test cases aimed to cover in the most
	efficient manner possible the Teaching Assistant Management System (TAMAS) we are to deliver. Following the
	paradigm of Test-Driven Development, we shall also implement the unit test cases. This will help
	keep the development of the system on track with the requirements imposed to the team. \\\\
	To produce high confidence in test results, it is important that the unit tests are designed in
	a systematic manner that organizes the test cases in such a way that it is clear that all
	branches in logic are covered. To achieve this, the technique of equivalence partitioning will
	be exploited. This involves dissecting each test subject into its equivalent input and output
	partitions, which gives a great foundation concerning how and which unit tests should be
	written.\\\\
	It is when the system requires new functionality that the writing and running of
	tests will be triggered. These tests are designed to fail at first, since they will be written
	before the code which they attempt to test, as per the philosophy of the Test Driven Development
	paradigm. The new code will be implemented in order to ensure that these tests pass; further
	following the
	Test Driven Development framework. Another trigger for the execution of tests is to reach full
	statement and branch coverage for the system if it is not already so. Finally, regular
	systematic tests will be executed to ensure that no editing has caused any unforeseen bugs.
	
\section{Test Cases}
    All the classes to test will have the following pattern:\\
    
    \textbf{CLASS TO TEST/NOT TO TEST}
    \begin{itemize}
        \item Reason to test/not to test
        \item attributes and related tests
    \end{itemize}
    
    \subsection{Classes To Test}
    \begin{itemize}
        \item \textbf{Course}
        \begin{itemize}
            \item \textbf{Reason to test:} This class is intrinsic to the job posting and application transaction. Its attributes TA/Grader time budget define how many students can apply to the position. Less relative to the application, but still needed for the correct behavior of the process is the fact that the student must have experience in the given course in order to apply for a position as TA/Grader for it.
            \item \textbf{Test Cases:} 
            \begin{itemize}
                \item \textbf{className}: test for null, empty, only numerical and only spaces inputs.
                \item \textbf{CDN}: test for non-unique, null, empty, not alphabetic and only spaces inputs.
                \item \textbf{className}: test for null, empty, not numerical and only spaces inputs.
                \item \textbf{graderTimeBudget/taTimeBudget}: test for null, empty, not numerical and negative inputs.
                \item \textbf{addJob:} test for null, and empty inputs for addJob. test for Time inputs for startTime and endTime, strictly alphabetical input for Day, and Requirements, default instructor input for Instructor, and strictly numerical input for salary.
                \item \textbf{addJobAt and addOrMoveJobAt:} test for adding job at a null position, moving existing Job at an incrementally higher and at an incrementally smaller position than it is already at, and at a position that is already occupied.
                \item \textbf{Retrieval:} test for successful retrieval of non-null className, CDN, number of and minimum number of jobs, Grader Time Budget, Job, List of Job objects, and Ta Time Budget.
                \item \textbf{Delete and Removal:} test for successful and failure of removal of a Job object, and successful and failure of deletion of a Course object.
            \end{itemize}
        \end{itemize}
        
        \item \textbf{Job}
        \begin{itemize}
            \item \textbf{Reason to test:} This class is intrinsic to the job posting and application transaction. It encapsulates half of the persistence aspect of the app which is to publish and view job postings.
            \item \textbf{Test Cases:} 
            \begin{itemize}
                \item \textbf{CDN}: test for non-existent, null, empty, only numerical and only spaces inputs.
                \item \textbf{requirements}: test for null, empty, empty and only spaces inputs.
                \item \textbf{position}: test for null inputs.
                \item \textbf{salary}: test for null, empty, not numerical and negative inputs.
                \item \textbf{day}: test for null, empty and weekend inputs.
               \item \textbf{startTime/endTime}: test for null, empty and outside of working hours inputs.
               \item \textbf{Retrieval:} test for successful retrieval for all getter methods.
            \end{itemize}
        \end{itemize}
        
        \item \textbf{Tutorial}
        \begin{itemize}
            \item \textbf{Reason to test:} This class is intrinsic to the job posting and application transaction. Job postings are for either TA/Grader. In the first case, the availabilities of the TA must be compatible with those of the tutorial.
            \item \textbf{Test Cases:} 
            \begin{itemize}
                \item test the same attributes as Course to ensure inheritance was correctly implemented by the UML
            \end{itemize}
        \end{itemize}
        
        \item \textbf{Laboratory}
        \begin{itemize}
            \item \textbf{Reason to test:} This class is intrinsic to the job posting and application transaction. Job postings are for either TA/Grader. In the first case, the availabilities of the TA must be compatible with those of the laboratory.
            \item \textbf{Test Cases:} 
            \begin{itemize}
                \item test the same attributes as Course to ensure inheritance was correctly implemented by the UML
            \end{itemize}
        \end{itemize}
        
        \item \textbf{Profile}
        \begin{itemize}
            \item \textbf{Reason to test:} This class is intrinsic to the job posting and application transaction. The system requires profile identifiers to discern which methods and attributes each instance has access to.
            \item \textbf{Test Cases:} 
            \begin{itemize}
                \item \textbf{id}:test for non-unique, null, empty, not alphabetic and only spaces inputs.
                \item \textbf{username}: test for non-unique, null, empty, not alphanumeric and only spaces inputs.
                \item \textbf{password}: test for invalid difficulty, null and empty inputs.
                \item \textbf{firstName/lastName}: test for null, empty, not alphabetical and only spaces inputs.
            \end{itemize}
        \end{itemize} 
        
        \item \textbf{Student}
        \begin{itemize}
            \item \textbf{Reason to test:} This class is intrinsic to the job posting and application transaction. The student is the only instance of profile allowed to apply to a job posting.
            \item \textbf{Test Cases:} 
            \begin{itemize}
                \item test the same attributes as Profile to ensure inheritance was correctly implemented by the UML
                \item \textbf{experience}: test for null, empty and only spaces inputs.
                \item \textbf{degree}: test for null inputs.
                \item \textbf{Jobs}: test for retrieval of job list
            \end{itemize}
        \end{itemize}
        
        \item \textbf{Instructor}
        \begin{itemize}
            \item \textbf{Reason to test:} This class is intrinsic to the job posting and application transaction. The instructor is the only instance of profile allowed to post jobs.
            \item \textbf{Test Cases:} 
            \begin{itemize}
                \item test the same attributes as Profile to ensure inheritance was correctly implemented by the UML
                \item \textbf{experience}: test for null, empty and only spaces inputs.
                \item \textbf{degree}: test for null inputs.
                \item \textbf{Jobs}: test for retrieval and modification of job list
                \item \textbf{Application}: test for retrieval of application list
                \item \textbf{course}: test for retrieval of course list
            \end{itemize}
        \end{itemize}
        
        	\item \textbf{Application}
        	\begin{itemize}
        		\item \textbf{Reason to test:} This class is necessary for the Student to apply for a job. Each application is associated to a student and a job posting; hence one needs to test for different Student attributes for different Jobs.
        		\item \textbf{Test Cases:} 
        		\begin{itemize}
        			\item \textbf{Test Apply for a Job:} test for error if hours are not compatible, test for error if job is null, and test for error if student is null.
        		\end{itemize}
        		
        	\end{itemize}
        \end{itemize}
        
        \begin{itemize}
        	\item \textbf{Admin}
        	\begin{itemize}
        		\item \textbf{Reason to test:} This class is intrinsic to the creation of a course, the creation of a Student and Instructor, and application transaction. Its attributes inherit from profile. They define information about the person that is registering an admin profile; hence it is needed to test for different attribute inputs and for successful creation of classes.
        		\item \textbf{Test Cases:} 
        		\begin{itemize}
        			\item test the same attributes as Profile to ensure inheritance was correctly implemented by the UML
        			\item \textbf{Application Transaction:} test for successful application transaction between Student and Job Posting
        			\item \textbf{Successfully Create Classes:} test to successfully create Instructor, and Student.
        		\end{itemize}
        		
        	\end{itemize}
        \end{itemize}
        
        \begin{itemize}
        \item \textbf{Persistence}
        \begin{itemize}
            \item \textbf{Reason to test:} This class is intrinsic to the job posting and application transaction. Without persistence capability the controllers have no data from which to derive the desired outcome
            \item \textbf{Test Cases:} 
            \begin{itemize}
                \item creation, modification and deletion of Course, Job, Application and Profile instances. 
            \end{itemize}
        \end{itemize}
        
    \end{itemize}
    
    \subsection{Classes to Not Test}
    \begin{itemize}
    	\item \textbf{Application Manager}
    	\begin{itemize}
    		\item \textbf{Reason to not test:} This class acts as a container for Application and Job classes. There is no unit test to be done on the ApplicationManager class. Testing for persistence of this manager class and its relation to Application and Job are not to be done with unit tests.
    	\end{itemize}
    \end{itemize}
    
    \begin{itemize}
    	\item \textbf{Profile Manager}
    	\begin{itemize}
    		\item \textbf{Reason to not test:} This class acts as a container for Student, Instructor and Admin classes. There is no unit test to be done on the ProfileManager class. Testing for persistence of this manager class and its relation to Student, Instructor and Admin are not to be done with unit tests.
    	\end{itemize}
    \end{itemize}
    
    \begin{itemize}
    	\item \textbf{Course Manager}
    	\begin{itemize}
    		\item \textbf{Reason to not test:} This class acts as a container for Course class. There is no unit test to be done on the CourseManager class. Testing for persistence of this manager class and its relation to Course are not to be done with unit tests.
    	\end{itemize}
    \end{itemize}
    
    \section{Techniques and Tools}
    
    The system is to be designed in a complete Test-Driven Development paradigm. Hence all test
    cases will be written before the actual controllers are implemented. Furthermore, the order in
    which the tests will be passed follows the same order as the requirements derived in the
    Requirements document of deliverable \#1. This will aid the classification and prioritization of
    each bi-weekly runs' objectives. In parallel to this method, the team will rely heavily on code
    revisions by the senior members of the team to ensure the code written is up to visual,
    performance and logical standards. In terms of frequency, individual nightly tests will be ran
    on all platforms of the system and bi-weekly builds and test will be done every Saturday to
    ensure the deliverable validates and verifies the given requirements. To facilitate the testing
    process, the team will use the following tools:
    
    \begin{itemize}
    	\item \textbf{Unit Test Framework:} Automated, well documented and supported system to allow a standardized set of tests to be done regardless of the platform. JUnit and PHPUnit will be used.
    	\item \textbf{Test Coverage Tool:} Ensure that the measure to which the tests and actions cover the source code is optimal by function, statement, branch and condition standards. Its outputs will be used during the code revision sessions. The following tools will be used for these purposes. The tools used for coverage analysis include: EclEmma for Eclipse for the Desktop/Laptop Platform, jacoco Gradle plugin for the mobile platform, and PHPUnit for the web platform.
    \end{itemize}
    
    
    \section{Differences in Unit Testing for the Three Platforms}
    
    The differences when it comes to unit testing for the three platforms are: the classes that
    are to be tested, the tools that are to be used for coverage analysis, and the test unit libraries that
    are to be used.
    The desktop/laptop app has all functionality; hence, all unit tests for all of the classes to be
    tested will be at least done on this platform. The mobile and web apps have less functionality
    than the desktop/laptop app; hence, there will be less classes to be tested on these two
    platforms. 
    The JUnit test library will be used for the testing of the desktop/laptop and the mobile platforms. The
    PHPUnit library and tools will be used for the testing of the web platform. Their coverage analysis tools are also
    different and each platform's specific coverage tool is mentioned above in Test Coverage Tool.
    The mobile app is implemented for the student's use; hence, there will be no
    unit testing for the Admin, Instructor, and Course classes on this platform. More specifically,
    the plan of unit testing the mobile app is to unit test the Student, Application, and Job
    classes. 
    The web app is implemented for the instructor's use; hence, there will be no unit testing for
    the Admin, and Student classes on this platform. More specifically, the plan of unit testing the
    web app is to unit test the Instructor, Application, Job, and Course classes.
    
    \section{Coverage Statistics}
    
    Since the plan is to implement certain functionalities in a Test Driven Development paradigm, it
    will be necessary to execute tests before there is even code written for the test. Tests will be
    executed often during the development of new functionality and during the verification and
    validation of existing functionality. In other words, JUnit tests will be written and executed
    before a certain method is written. JUnit tests will be written and executed shortly after test
    cases are made for a certain method that is already implemented. 
    
    Thus, the goal of the unit tests is to ensure that each minimal unit of the system is
    functioning correctly, so that these units may serve as a solid foundation for the system.
    Referring to the rationale of the test plan, it is clear that each class is tested
    independently, as well as each method within the class. The main goal behind this strategy is to
    isolate every area where workflow may lead to different logical steps in the methods, and to
    attempt to examine the behavior at each of these workflows.\\\\
    So, based on the structure of the unit test plan, it is clear that each logical path of each
    method is being examined. The technique of equivalence partitioning is used to organize all
    possible equivalent input and output patterns, such that it can be assured that each one is
    verified accordingly. It can then be concluded that all logical
    branches of the code have been covered. Of course, as the system evolves, the unit test plan may
    need to evolve as well for such coverage to be attained. Furthermore, until the system is
    written to more completion, it will be very difficult to ensure that all branches are guaranteed
    to be covered. As the system develops, using tools such as EclEmma will aid in discovering holes
    in branch and statement coverage.\\\\
    As the system currently stands, however, it is believed that the unit test plan above will cover
    approximately 100\% of the branches. It is very difficult to predict state coverage statistics
    at such an early stage in development, but the goal is to have similar statistics for state
    coverage as well. Given the 9 classes to test in the Unit Testing Plan, encompassing 34 test
    cases that must deal with various different input patterns, it is not unrealistic that 100\% of
    branches, as well as statements, can be verified by this team.
    
\part{Integration Test Plan}
\section{Description}
\textbf{The integration testing is meant to test the relationships between units that form
subsystems. The aim is to test all the relationships between classes that form subsystems. Since the
Controller classes control all the interactions between classes, they are the classes that will be tested in
the integration testing.}
%%%%%%%%%%%% ADD INTEGRATION TEST PLAN HERE%%%%%%%%%%%%%%%%%%%%%
\section{Integration Strategy}

The strategy of integration will follow a bottom-up approach. The system will first have all the
unit tests done for each class. Then the controller classes will integrate the units together into
discrete subsystems. Since course controller is going to be used on all platforms, it was decided to
test this subsystem first. The other subsystems can be tested in parallel on their individual
platforms.
 \subsection{Subsystem Testing Diagram}
	The following diagram shows the subsystems and interactions that will be tested. The arrows show
	how each of the orange units relate to the controller subsystems. 
 \begin{figure}[H]
    \centering
    \includegraphics[scale = 0.65]{IntegrationDiagram.png}
 \end{figure}
 \subsection{Platform Integration}
 
Since our mobile app is only for students, there is no need to test the course controller on the mobile
application. Likewise, the web app will not need to test the application controller since the web app
is just for teachers. All the tests that follow are done programmatically rather than through a user
interface. For example, Applying for a job will involve just creating an instance of an application
and attributing it a profile, a job, etc. The system test will test whether the user can create a
form and apply through the mobile application, for example.
 
\section{Integration Test descriptions}
\subsection{Subsystem Testing}
\begin{itemize}
    \item \textbf{Profile Persistence Subsystem Testing}
         \begin{itemize}
			 \item \textbf{Reason to test:} These tests will evaluate the relationship between the
				 profiles the persistence layer, this is necessary to ensure that all profile data
				 and associations are properly propagated throughout the system for later use.
             \item \textbf{Situations:}
             \begin{itemize}
				 \item \textbf{Create a profile:} Test to ensure that a profile can properly be
					 created in persistence.
                 \begin{itemize}
					 \item Create profile using complete valid profile data (including the courses
						 the profile is involved in) and ensure that it is properly persisted and
						 available to other classes.
					 \item Attempt to create a profile using invalid or incomplete input data,
						 ensure that it is not propagated in the persistence layer.
					 \item Attempt to create a profile by giving no input data, ensure that this is
						 not propagated in the persistence layer.
                 \end{itemize}
                 \item \textbf{Modify a profile:}
                 \begin{itemize}
					 \item Modify a profile so that the new profile data is valid, this new profile
						 data should persist.
					 \item Modify a profile so that the new data is invalid/incomplete, the system
						 should not allow this so the persistence layer will remain unchanged.
                 \end{itemize}
                 \item \textbf{Delete a profile:}
                 \begin{itemize}
                     \item Deleting a valid profile should remove it's data from the persistence layer.
					 \item Deleting an invalid/nonexistent profile should be rejected and have no
						 effect on the persistence layer.
                 \end{itemize}
             \end{itemize}
         \end{itemize}
    \item \textbf{Course Persistence Subsystem Testing}
         \begin{itemize}
			 \item \textbf{Reason to test:} These tests will evaluate the relationship between the
				 courses, the profiles and the persistence layer this is necessary to ensure that
				 all course data and associations are properly propagated throughout the system.
             \item \textbf{Situations:}
             \begin{itemize}
                 \item \textbf{Create a course} 
                 \begin{itemize}
                     \item Create a course using complete valid course data and ensure that it is properly persisted and available to other classes.
                     \item Attempt to create a course using invalid or incomplete input data, ensure that it is not propagated in the persistence layer.
                     \item Attempt to create a course by giving no input data, ensure that this is not propagated in the persistence layer.
                 \end{itemize}
                 \item \textbf{Modify a course}
                \begin{itemize}
                     \item Modify a course so that the new data is valid, this new course data should persist.
                     \item Modify a  course so that the new data is invalid/incomplete, the system should not allow this so the persistence layer will remain unchanged.
                 \end{itemize}
                 \item \textbf{Delete a course}
                \begin{itemize}
                    \item Deleting a valid course should remove it's data from the persistence layer.
                     \item Deleting an invalid/nonexistent course should be rejected and have no effect on the persistence layer.
                 \end{itemize}

             \end{itemize}
         \end{itemize}

    \item \textbf{Application Persistence Subsystem Testing}
         \begin{itemize}
             \item \textbf{Reason to test:} These tests will evaluate the relationship between the profiles and the jobs that make up the applications and the persistence layer.
             \item \textbf{Situations:}
             \begin{itemize}
                 \item \textbf{Create an application}
                 \begin{itemize}
                     \item Create an application using complete valid data and ensure that it is properly persisted and available to other classes.
                     \item Attempt to create an application using invalid or incomplete input data, ensure that it is not propagated in the persistence layer.
                     \item Attempt to create an application by giving no input data, ensure that this is not propagated in the persistence layer.
                 \end{itemize}
                 \item \textbf{Modify an application}
                 \begin{itemize}
                     \item Modify an application so that the new data is valid, this new course data should persist.
                     \item Modify an application so that the new data is invalid/incomplete, the system should not allow this so the persistence layer will remain unchanged.
                 \end{itemize}
                 \item \textbf{Delete an application}
                 \begin{itemize}
                    \item Deleting a valid application should remove it's data from the persistence layer.
                     \item Deleting an invalid/nonexistent application should be rejected and have no effect on the persistence layer.
                 \end{itemize}
             \end{itemize}
         \end{itemize}
\end{itemize}     

\section{Techniques and Tools}
Using EclEmma, we aspire to have near 100\% code coverage. This would imply that there should be near 100\% statement and branch coverage. The goal is to test the relationship between all the classes.

\section{Coverage Statistics}
Since there are 3 classes in each subsystem, and the tests should cover all the relationships
between the classes, there should be 3! or 6 relationships in each subsytem totaling to 18
relationships. It was decided that it would be more efficient not to create a test for every
relationship, but to instead test functionality that includes all these relationships within the
tests. For example, creating a profile would test the persistence layer as well as the profile class
together. When the profile class is attributed a course within this test, it tests the integration
between all three classes. For this reason, we aim for 100\% code coverage.

%%%%%%%%%%%%%%%%%%%%%%%%%%%%%%%%%%%%%%%%%%%%%%%%%%%%%%%%%%%%%%%%%
\part{System Test Plan}
\section{Description}
The purpose of system testing is to validate the functionality of the system as a whole. System
tests are black-box tests, meaning they are done without any concern regarding the code - that is to
say, only the inputs and outputs are examined.\\\\
So as to examine the functionality of the system in an organized and systematic fashion, system
tests will be designed according to the various \textit{use cases} of the system. The use cases
provide concrete details concerning how each functionality of the system should behave, and in
particular how each functionality of the system should react to unexpected or faulty input.\\\\
The proposed system tests will be discussed below.
\subsection{Administrator-Specific Actions}
\begin{enumerate}
	\item \textbf{Sending Student Enrollment Data and Course List}
		\begin{enumerate}
			\item Valid data entered
			\begin{itemize}
				\item Enter appropriate data for student enrollment, as well as the course list, in the
					desktop application. Ensure that the proper XML files are generated.
			\end{itemize}
			\item Invalid student enrollment data entered
			\begin{itemize}
				\item Enter faulty student enrollment data (by omitting names, for example) and
					enter correct course list data in the desktop application. Ensure that an error
					message about invalid student data is made present. Moreover, assert that the
					corresponding student enrollment XML has not been generated at all, and that the
					course list XML file has been generated appropriately.
			\end{itemize}
			\item Invalid course list data entered
			\begin{itemize}
				\item Enter faulty course list data (by omitting required names, or by including
					non-integral course codes, for example) and enter correct student data in the
					desktop application. Ensure that an error message about invalid course list data
					is made present. Moreover, assert that the corresponding course list XML file
					has not been generated at all, and that the student enrollment XML file has been
					generated appropriately.
			\end{itemize}
		\end{enumerate}
	\item \textbf{Generating Initial Student Job Placements}
		\begin{enumerate}
			\item Valid XML Application file is present
				\begin{itemize}
					\item Ensure that the appropriate XML file containing data about student
						applications is present. Use desktop application to generate default TA
						placement.
						\begin{enumerate}
							\item Verify that an appropriate XML file containing data about the job
								offerings was generated.
							\item Verify that initial placement confines to the budget for each
								department.
							\item Verify that graduate students were selected with higher priority
								than undergraduate students.
							\item Verify that the same student was selected for as many jobs as
								possible for the same course.
						\end{enumerate}
				\end{itemize}
			\item XML Application file not found or is corrupted
				\begin{itemize}
					\item Place invalid XML file (containing syntax errors or type mismatches, for
						example) in the appropriate location, and attempt to generate default TA
						placement with the desktop application. Ensure that an error message about
						the invalidity of the XML file is output, and that no XML file about job
						offerings has been created.
				\end{itemize}
		\end{enumerate}
	\item \textbf{Accepting or Rejecting Instructor TA Modifications}
		\begin{enumerate}
			\item Accepting a modification
				\begin{itemize}
					\item Ensure that a list of all instructor modifications is present in the
						\textit{view proposed modifications} page. Select a modification for
						viewing, and select to accept the modification.
						\begin{enumerate}
							\item Ensure that the original allocations XML file has been overwritten
								using the data from the modification XML file that is currently
								active.
						\end{enumerate}
				\end{itemize}
			\item Rejecting a modification
				\begin{itemize}
					\item Ensure that a list of all instructor modifications is present in the
						\textit{view proposed modifications} page. Select a modification for
						viewing, and choose to reject the modification.
						\begin{enumerate}
							\item Ensure that the original allocations XML file has not been
								modified.
							\item Ensure that the modifications XML file has been deleted.
						\end{enumerate}
				\end{itemize}
		\end{enumerate}
	\item \textbf{Send Job Offers}
		\begin{enumerate}
			\item Modification XML files present
				\begin{itemize}
					\item Ensure that a TA placement modification XML file is present. Attempt
						sending job offers in the \textit{view allocation} page.
						\begin{enumerate}
							\item Ensure that no \texttt{offers.xml} file is generated.
							\item Ensure that an error message informing the user that there are
								still outstanding modifications is output.
						\end{enumerate}
				\end{itemize}
			\item No modification XML files are present
				\begin{itemize}
					\item Remove all modification XML files, and ensure that initial allocation XML
						file is present. Attempt sending job offers in the \textit{view allocation}
						page.
						\begin{enumerate}
							\item Ensure that the \texttt{offers.xml} file has been generated with
								appropriate job offer data corresponding to the TA allocation.
						\end{enumerate}
				\end{itemize}
		\end{enumerate}
\end{enumerate}
\subsection{Instructor Use-Cases}
\begin{enumerate}
	\item \textbf{Publish Job Posting}
		\begin{enumerate}
			\item No course data found
				\begin{itemize}
					\item Remove all course data, or delete the \texttt{courses.xml} file. Attempt
						to open the page containing the \textit{publish job posting} form. Ensure
						that the page does not open, and an error message informing the user that no
						courses are available is output.
				\end{itemize}
			\item Course and Instructor data present
				\begin{itemize}
					\item Ensure the presence of the \texttt{courses.xml} and \texttt{profiles.xml}
						files, and make sure at least one course is present. Then, open the
						\textit{publish job posting} form and assert that for the indicated
						Instructor, all of its courses, and only courses that are being taught by
						that given professor, are available for selection in the Course list.
				\end{itemize}
			\item Valid input entered, no previous XML file found
				\begin{itemize}
					\item Specify valid information corresponding to a job posting in the
						\textit{publish job posting} form, and submit the data. Ensure that a XML
						file called \texttt{applications.xml} was generated and contains correct
						data concerning the job postings that were submitted.
				\end{itemize}
			\item Valid input entered, with previous XML file in place
				\begin{itemize}
					\item With a previously-generated \texttt{applications.xml} in place, attempt
						submitting a job posting using the \textit{publish job posting} form. Ensure
						that the XML file has been modified such that the new job postings have been
						appended after the previous ones (therefore, ensure no data was lost).
				\end{itemize}
			\item Type mismatches or empty fields submitted
				\begin{itemize}
					\item Attempt submitting a job posting, leaving fields blank or including type
						mismatches, such as a non-integral input in the Salary field. Ensure that
						the \texttt{applications.xml} file was not modified (and if it did not
						previously exist, ensure that it has not been generated).
				\end{itemize}
			\item Valid, but unsound data submitted
				\begin{itemize}
					\item Attempt submitting a job posting with unsound data, such as a start time
						that is later than an end time.
						\begin{enumerate}
							\item Ensure that the appropriate error message is output.
							\item Ensure that the \texttt{applications.xml} file was not modified
								(and if it did not previously exist, ensure that it has not been
								generated).
						\end{enumerate}
				\end{itemize}
		\end{enumerate}
	\item \textbf{Modify initial TA allocation}
		\begin{enumerate}
			\item No initial allocation data found
				\begin{itemize}
					\item Remove XML file with initial allocation data. Attempt to open TA placement
						modification page. Ensure that the page does not open, and that an error
						message informing the user that no allocation has been made is output.
				\end{itemize}
			\item Valid modifications entered
				\begin{itemize}
					\item Enter valid modifications in the modification form, and submit.
						\begin{enumerate}
							\item Ensure that the initial allocation XML file has not been
								modified.
							\item Ensure that a separate XML file containing the data about the
								modifications has been created appropriately.
						\end{enumerate}
				\end{itemize}
			\item Invalid modifications entered
				\begin{itemize}
					\item Enter modifications that have empty fields or type mismatches, for
						example. Attempt submitting modifications.
						\begin{enumerate}
							\item Verify that an error message containing the details of the
								invalidity of the input has been shown.
							\item Ensure that the initial allocation XML file has not been modified.
							\item Ensure that no separate XML files containing modification data
								have been generated.
						\end{enumerate}
				\end{itemize}
		\end{enumerate}
	\item \textbf{Evaluate TA/Grader}
		\begin{enumerate}
			\item No student XML data found or no TA's associated to current professor.
				\begin{itemize}
					\item Remove all student data, or remove job data for a specific professor.
						Under such professor's account, attempt to open the \textit{evaluate TA}
						page. Ensure that the page does not open, and that a message informing the
						user that no TA's are available is output.
				\end{itemize}
			\item TA data available
				\begin{itemize}
					\item Open the \textit{evaluate TA} page.
						\begin{enumerate}
							\item Ensure that only the TA's that worked for courses taught by the
								selected professor are available for evaluation.
							\item After submission, ensure that the Student that was evaluated had
								its \texttt{experience} field modified in the \texttt{profiles.xml}
								file.
						\end{enumerate}
				\end{itemize}
		\end{enumerate}
\end{enumerate}
\subsection{Student Use-Cases}
\begin{enumerate}
	\item \textbf{Apply to Job Posting}
		\begin{enumerate}
			\item No Job data available
			\begin{itemize}
				\item Remove \texttt{applications.xml} file, or ensure that it contains no Job data.
					Attempt opening the \textit{apply to job posting} page. Ensure that the page
					does not open and that a message informing the user that Jobs have not been
					created yet is output.
			\end{itemize}
			\item Valid input entered
			\begin{itemize}
				\item Ensure that Job data is available. Open \textit{apply to job posting} page and
					select a Job to apply to. Ensure that all Jobs that were present in the XML data
					are available in the selection list. Ensure that after submission, the
					\texttt{applications.xml} file was modified appropriately, and that no data has
					been lost.
			\end{itemize}
		\item Invalid input submitted
			\begin{itemize}
				\item Ensure that Job data is available. Open the \textit{apply to job posting} page
					and select a Job to apply to. Leave Job field blank, and attempt submitting.\
					\begin{enumerate}
						\item Ensure that an error message informing the user that no Job was
							selected has been output.
						\item Ensure that the \texttt{applications.xml} file has not been modified.
					\end{enumerate}
			\end{itemize}
		\end{enumerate}
	\item \textbf{Submit Personal Details}
		\begin{enumerate}
			\item Open \textit{modify profile} page. Ensure that all current data for the current
				Student is present in the form. Modify data in the forms and submit. Ensure that the
				\textit{profiles.xml} file has been modified appropriately.
		\end{enumerate}
	\item \textbf{Accept or Reject Job Offerings}
		\begin{enumerate}
			\item Open the \textit{job offers} page. Ensure that only jobs offered to the current
				Student are listed. Attempt accepting a job offer and rejecting a job offer. Ensure
				that the \texttt{offers.xml} file has been modified accordingly.
		\end{enumerate}
\end{enumerate}
\section{Rationale}
The system tests that were designed, and their classifications by main actors, were done to provide
an organized scheme to verify that all functionalities of the system are covered by the system
tests. Given the fact that system tests should be black-box, it only remains to determine if the
desired use cases of the system are functioning according to the client's requirements.\\\\
To ensure that all requirements are met, the system tests were divided into categories specified by
the actor that would use those functionalities. Section \textbf{10.1} shows system tests associated
with the functionalities that only an Administrator would be responsible for. Then, section 
\textbf{10.2} shows system tests associated with the Instructor's actions. Finally, section
\textbf{10.3} shows system tests associated with the Student's actions. This categorization made it
much simpler to verify that all functionalities were tested, because it allowed functionalities to
be listed off as the Use Cases defined in the first Deliverable.\\\\
It is important to note that all logic-related bugs concerning the fine details of the calculations
carried out by the system should be already covered by the Unit Tests and Integration tests above.
Hence, the goal of the System Test plan is to design a routine that tests the communications and
interactions between subsystems that were tested in the Integration Tests. Such communications
involve mainly the flow of information from the View classes all the way down to the Controller
classes that store information about the domain entities in the persistence. All tests described in
section \textbf{10} verify the proper functioning of the input parsing from the Views, and the output
results of the persistence. Once again, everything that happens between the input and the output is
in a \textit{black-box}, and it is assumed to have been tested in the previous testing schemes.\\\\
Another important thing to consider when designing these System Tests is that the users will not
always enter valid inputs, and illegal actions may be attempted. Therefore, testing the ideal inputs
only is not a good idea, because it is equally important that the system handles erroneous or
invalid input properly as well. To take this into account, each System Test Case was examined, and
possible areas of invalid input were found. Each of those invalid input scenarios is accounted for
in the System Test plan. Most importantly, it is imperative that submitting erroneous input does not
affect the current persistence. Although each test case may experience significantly different
errors, if the persistence is not modified under these conditions, the errors will not affect the
\textit{bigger picture} of the whole system. Therefore, to complete each test case, any possible
invalid inputs are isolated, and it is verified that the persistence is not affected by them.\\\\
By carrying out the tests described in section \textbf{6}, it is ensured that all Use Cases have
been verified, and all bad behavior on the user's end can be handled reliably. Therefore, assuming
all tests succeed, it can be confided that all user requirements
are functioning properly. As the project continues, new System Tests may be implemented to reflect
changes in the user requirements.
\section{Coverage Statistics}
Given the black-box nature of system testing, the focus should be more on the fact
that all branches of logic are tested rather than all statements individually.\\\\
Splitting the global actions and logical sequences by platform we have:
\begin{itemize}
    \item \textbf{Desktop (Admin):}
    \begin{itemize}
		\item Create/Update all Profiles, Jobs and Applications $\rightarrow$ Tested in section \textbf{10.1-3}
        \item Publish a job posting as any given Instructor $\rightarrow$ Tested in section \textbf{10.1-2} and \textbf{10.1-4}
        \item Apply to a job posting as any given Student $\rightarrow$ Tested in section \textbf{10.1-1} 
		\item View and modify attributes of all available Courses $\rightarrow$ Tested in section
			\textbf{10.1-1}
		\item View, modify, and reject/accept Instructor modifications to the TA allocations
			$\rightarrow$ Tested in section \textbf{10.1-3}.
    \end{itemize}
    \item \textbf{Web (Instructor):}
        \begin{itemize}
        \item Create/Publish his own postings $\rightarrow$ Tested in section \textbf{10.2-1}
        \item View the attributes to its own available Courses and Profile $\rightarrow$ Tested in
			section \textbf{10.2-1}.
		\item View TA allocation and make modifications $\rightarrow$ Tested in section
			\textbf{10.2-2}.
		\item Send feedback for TA's that worked for their courses $\rightarrow$ Tested in section
			\textbf{10.2-3}.
    \end{itemize}
    \item \textbf{Mobile (Student):}
        \begin{itemize}
        \item Apply to the postings available to him $\rightarrow$ Tested in section \textbf{10.3-1}
		\item view the attributes to its own available Courses and Profile $\rightarrow$ Tested in
			section \textbf{10.3-2}
    \end{itemize}
\end{itemize}
Total number of logical actions: 16\\
Total number of logical actions tested: 16\\
Note that the logical actions tested consider the fact that the action was tested on a plethora of
edge conditions, such that all alternative workflows are tested. Refer to section \textbf{10} to
clear any skepticism.\\\\
From the thorough enumeration above, it is clear that all use cases of the system (as well as all of
their alternative workflows) are examined by the System Test Plan. Therefore, it is expected to have
close to 100\% test coverage at the system level. Of course, as time passes, there is always the
possibility that the client will wish to modify some of the requirements, at which point the System
Test Plan, as well as the coverage statistics, may change.

\section{Detailed Descriptions of Two Test Cases}
The following section will outline two system test cases in such great detail that they could be
implemented without the authors' assistance. Due to the omnicapable power of the desktop
application, the following two test cases will be described for testing on the desktop.
\subsection{The Publish Job Posting Test Case}
\subsubsection{Setting Up}
\begin{enumerate}
	\item Remove prior XML files. If the system is being run from Eclipse, remove all files in the
		\texttt{output/} directory.
	\item Open the main menu
	\item Select ``Publish Job Posting''
	\item \textbf{Verify that the Publish Job Posting Window does not appear, but rather an error
		message saying that no instructors or courses are available is displayed.}
\end{enumerate}
\subsubsection{Creating Instructor Profiles}
\begin{enumerate}
	\item Open the main menu
	\item Select ``Register Profile''
	\item Select the Instructor radio button
	\item In the First Name field, type ``Jim''
	\item In the Last Name field, type ``Lahey''
	\item In the Username field, type ``hwn1977''
	\item In the Password field, type ``pswd''
	\item Click ``Submit''. A confirmation message saying ``Instructor hwn1977 was created'' should
		appear, and all fields are reset to blank.
	\item Now, for First Name, Last Name, Username, and Password, enter ``Randy'', ``Bobandy'',
		``cheeseburger'',``pswd2'' respectively. Finally, press submit and close the Register
		Profile window.
	\item From the main menu, select ``Publish Job Posting''.
	\item \textbf{Verify that the Publish Job Posting Window does not appear, but rather an error
		message saying that no courses are available is displayed.}
\end{enumerate}
\subsubsection{Creating the Courses}
\begin{enumerate}
	\item Open the main menu
	\item Select ``Create Course''
	\item The names ``Jim Lahey'' and ``Randy Bobandy'' should now be seen in the list view at the
		top. Select Jim Lahey.
	\item In the Course Name field, type ``FACC100 - Trailer Park Supervision''
	\item In the CDN field, type ``1''
	\item In the Grader Time Budget field, enter ``19000.0''
	\item In the TA Time budget field, enter ``14000.0''
	\item Click ``Create Course''. All data in the fields will be cleared.
	\item With Jim Lahey still selected, enter the values ``ECSE322 - Fixing Lawnmowers'', ``2'',
		``20000.0,'', ``15000.0'' into the course name, CDN, grader budget, and TA budget fields,
		respectively. Click submit.
	\item Next, select Randy Bobandy from the list at the top of the page.
	\item Enter the values ``ECSE211 - Dealing with Abuse From an Old Drunk Man'', ``3'',
		``13'',``8.75'' into the Course Name, CDN, grader budget, and TA budget fields,
		respectively.
	\item Click submit and close the Create Course menu.
\end{enumerate}
At this stage, enough data has been created to test the Publish Job Posting Use Case. The XML files
for profiles and courses should already be in the appropriate location.
\subsubsection{Publishing a Job Posting}
\begin{enumerate}
	\item Open the main menu, and select ``Publish Job Posting''
	\item The names Jim Lahey and Randy Bobandy should be in the top-most list view. Select Jim
		Lahey.
	\item \textbf{Verify that ``FACC100 - Trailer Park Supervision'' and ``ECSE322 - Fixing
			Lawnmowers'' (and no other courses) appear in the list beneath the instructor names.}
	\item Select FACC100 - Trailer Park Supervision. Click ``Publish Job''.
	\item \textbf{Verify that an appropriate error message is displayed.}
	\item \textbf{Verify that, if there is already a \texttt{applications.xml} file, there is no
	mention of an application for Job with the current course name.}
	\item Enter ``10.5'' in the salary field, and ``Associate Trailer Park Supervision Experience''
		in the requirements field. Then, set the end time back by two hours so it is before the
		start time. Click ``Publish Job''.
	\item \textbf{Verify that an appropriate error message is displayed.}
	\item \textbf{Verify that, if there is already a \texttt{applications.xml} file, there is no
	mention of an Job for the course chosen.}
	\item Adjust the end time to be after the start time. Press ``Publish Job''.
	\item \textbf{Verify that the \texttt{applications.xml} file contains the data for a Job for
		this chosen course. If other Jobs were posted prior to this one, verify that their
	corresponding data remains in the XML file.}
	\item Chose Randy Bobandy from the top-most list. \textbf{Verify that ECSE211 - Dealing with
		Abuse From an Old Drunk Man is the only option in the list below.} 
	\item Select the ECSE211 option. Click ``Publish Job''.
	\item Repeat steps 6-11.
\end{enumerate}
\subsection{Applying for a Job}
\subsubsection{Setting up}
\begin{enumerate}
	\item Remove prior XML files. If the system is being run from Eclipse, remove all files in the
		\texttt{output/} directory.
	\item Open the main menu
	\item Select ``Create Job Application''
	\item \textbf{Verify that the Job Application window does not appear, but rather an error
		message saying that no jobs or students are available is displayed.}
	\item Repeat the \textit{Creating Instructor Profiles} and \textit{Creating Courses} steps from
		section \textbf{13.1}.
	\item Repeat steps 2-3.
	\item Repeat the \textit{Publishing a Job Posting} steps from section \textbf{13.1}, and repeat
		steps 2-3.
\end{enumerate}
\subsubsection{Creating Students}
\begin{enumerate}
	\item Open the main menu. Select ``Register Profile''
	\item Select the Student radio button.
	\item Enter the values ``Cosmo'', ``Kramer'', ``kman'', ``pennypacker'' into the First Name,
		Last Name, Username, and Password fields, respectively.
	\item Enter ``Golfing'' in the Skills textbox.
	\item Click ``Submit''. A message saying ``Student kman created'' should be seen at the top of
		the window, and all fields should be reset.
	\item Enter ``George'', ``Costanza'', ``vandelay'',``bosco'' into the First Name, Last Name,
		Username, and Password fields, respectively.
	\item Enter ``Baseball'' in the Skills textbox.
	\item Click ``Submit''. A message saying ``Student vandelay created'' should be seen at the top
		of the window. Close the window.
\end{enumerate}
\subsubsection{Creating A Job Application}
\begin{enumerate}
	\item From the main menu, select ``Create a Job Application''. The Job Application window should
		finally instantiate.
	\item Two lists should be present, one containing the student names, and the other containing
		the job names. In future versions of the system, students will log in and will not need to
		select their name from a list. For this version, select any student from the list.
	\item Without selecting a Job from the list, click ``Apply''.
	\item \textbf{Verify that an error message saying that no Job was selected is present, and
			ensure that the \texttt{applications.xml} file has not been modified.}
	\item Select a Job from the list. Click ``Apply''. A friendly message wishing the given student
		luck for his application to the selected Job will appear.
	\item \textbf{Verify that the \texttt{applications.xml} file has included the job application.}
	\item Repeat steps 5 and 6 with different combinations of students and Jobs.
	\item \textbf{Verify that no data is lost in the \texttt{applications.xml} file when multiple
		applications are created. That is to say, make sure the file has persisted all applications and
	all courses after multiple applications were submitted.}
\end{enumerate}
\end{document}
