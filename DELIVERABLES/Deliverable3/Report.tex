\documentclass[12pt]{report}

\usepackage{amsmath,amssymb,amsfonts} %In case we need math stuff
\usepackage{graphicx} %For inserting images and stuff
\usepackage{listings} %For inserting code snippets
\usepackage{enumerate} %For fancy enumeration
\usepackage{hyperref}
\usepackage{pdfpages}
\usepackage{float}
\usepackage[margin=2cm]{geometry} %Nice margin setting

\renewcommand*\thesection{\arabic{section}}

\title{ECSE 321 - Intro to Software Engineering\\Design Specification Document - Deliverable 3}
\author{Harley Wiltzer\\Camilo Garcia La Rotta\\Jake Shnaidman\\Robert Attard\\Matthew Lesko}
\date{March 19, 2017}

\begin{document}
\pagenumbering{gobble} %No page number on title page
\maketitle
\newpage
\pagenumbering{arabic} %Arabic numeral page numbers on regular pages
\tableofcontents
\part{Unit Test Plan}
\section{Description}
	In this section we will present an extensive suite of test cases aimed to cover in the most
	efficient manner possible the Fantasy Basketball system we are to deliver. Following the
	paradigm of Test-Driven Development, we shall also implement the unit test cases. This will help
	keep the development of the system on track with the requirements imposed to the team. 
	
\section{Test Cases}
    All the classes to test will have the following pattern:\\
    
    \textbf{CLASS TO TEST/NOT TO TEST}
    \begin{itemize}
        \item Reason to test/not to test
        \item attributes and related tests
    \end{itemize}
    
    \subsection{Classes To Test}
    \begin{itemize}
        \item \textbf{Course}
        \begin{itemize}
            \item \textbf{Reason to test:} This class is intrinsic to the job posting and application transaction. Its attributes TA/Grader time budget define how many students can apply to the position. Less relative to the application, but still needed for the correct behaviour of the process is the fact that the student must have experience in the given course in order to apply for a position as TA/Grader for it.
            \item \textbf{Test Cases:} 
            \begin{itemize}
                \item \textbf{className}: test for null, empty, only numerical and only spaces inputs.
                \item \textbf{CDN}: test for non-unique, null, empty, not alphabetic and only spaces inputs.
                \item \textbf{className}: test for null, empty, not numerical and only spaces inputs.
                \item \textbf{graderTimeBudget/taTimeBudget}: test for null, empty, not numerical and negative inputs.
                \item \textbf{Jobs}: test for retrieval of job list
            \end{itemize}
        \end{itemize}
        
        \item \textbf{Job}
        \begin{itemize}
            \item \textbf{Reason to test:} This class is intrinsic to the job posting and application transaction. It encapsulates half of the persistence aspect of the app which is to publish and view job postings.
            \item \textbf{Test Cases:} 
            \begin{itemize}
                \item \textbf{CDN}: test for non-existent, null, empty, only numerical and only spaces inputs.
                \item \textbf{requirements}: test for null, empty, empty and only spaces inputs.
                \item \textbf{position}: test for null inputs.
                \item \textbf{salary}: test for null, empty, not numerical and negative inputs.
                \item \textbf{day}: test for null, empty and weekend inputs.
               \item \textbf{startTime/endTime}: test for null, empty and outside of working hours inputs.
            \end{itemize}
        \end{itemize}
        
        \item \textbf{Tutorial}
        \begin{itemize}
            \item \textbf{Reason to test:} This class is intrinsic to the job posting and application transaction. Job postings are for either TA/Grader. In the first case, the availabilities of the TA must be compatible with those of the tutorial.
            \item \textbf{Test Cases:} 
            \begin{itemize}
                \item test the same attributes as Course to ensure inheritance was correctly implemented by the UML
            \end{itemize}
        \end{itemize}
        
        \item \textbf{Laboratory}
        \begin{itemize}
            \item \textbf{Reason to test:} This class is intrinsic to the job posting and application transaction. Job postings are for either TA/Grader. In the first case, the availabilities of the TA must be compatible with those of the laboratory.
            \item \textbf{Test Cases:} 
            \begin{itemize}
                \item test the same attributes as Course to ensure inheritance was correctly implemented by the UML
            \end{itemize}
        \end{itemize}
        
        \item \textbf{Profile}
        \begin{itemize}
            \item \textbf{Reason to test:} This class is intrinsic to the job posting and application transaction. The system requires profile identifiers to discern which methods and attributes each instance has access to.
            \item \textbf{Test Cases:} 
            \begin{itemize}
                \item \textbf{id}:test for non-unique, null, empty, not alphabetic and only spaces inputs.
                \item \textbf{username}: test for non-unique, null, empty, not alphanumeric and only spaces inputs.
                \item \textbf{password}: test for invalid difficulty, null and empty inputs.
                \item \textbf{firstName/lastName}: test for null, empty, not alphabetical and only spaces inputs.
            \end{itemize}
        \end{itemize} 
        
        \item \textbf{Student}
        \begin{itemize}
            \item \textbf{Reason to test:} This class is intrinsic to the job posting and application transaction. The student is the only instance of profile allowed to apply to a job posting.
            \item \textbf{Test Cases:} 
            \begin{itemize}
                \item test the same attributes as Profile to ensure inheritance was correctly implemented by the UML
                \item \textbf{experience}: test for null, empty and only spaces inputs.
                \item \textbf{degree}: test for null inputs.
                \item \textbf{Jobs}: test for retrieval of job list
            \end{itemize}
        \end{itemize}
        
        \item \textbf{Instructor}
        \begin{itemize}
            \item \textbf{Reason to test:} This class is intrinsic to the job posting and application transaction. The instructor is the only instance of profile allowed to post jobs.
            \item \textbf{Test Cases:} 
            \begin{itemize}
                \item test the same attributes as Profile to ensure inheritance was correctly implemented by the UML
                \item \textbf{experience}: test for null, empty and only spaces inputs.
                \item \textbf{degree}: test for null inputs.
                \item \textbf{Jobs}: test for retrieval and modification of job list
                \item \textbf{Application}: test for retrieval of application list
                \item \textbf{course}: test for retrieval of course list
            \end{itemize}
        \end{itemize}
        
        \begin{itemize}
        	\item \textbf{Application}
        	\begin{itemize}
        		\item \textbf{Reason to test:} This class is necessary for the Student to apply for a job. Each application is associated to a student and a job posting; hence one needs to test for different Student attributes for different Jobs.
        		\item \textbf{Test Cases:} 
        		\begin{itemize}
        			\item \textbf{Test Apply for a Job:} test for error if hours are not compatible, test for error if job is null, and test for error if student is null.
        		\end{itemize}
        		
        	\end{itemize}
        \end{itemize}
        
        \begin{itemize}
        	\item \textbf{Admin}
        	\begin{itemize}
        		\item \textbf{Reason to test:} This class is intrinsic to the creation of a course, the creation of a Student and Instructor, and application transaction. Its attributes inherit from profile. They define information about the person that is registering an admin profile; hence it is needed to test for different attribute inputs and for successful creation of classes.
        		\item \textbf{Test Cases:} 
        		\begin{itemize}
        			\item test the same attributes as Profile to ensure inheritance was correctly implemented by the UML
        			\item \textbf{Application Transaction:} test for successful application transaction between Student and Job Posting
        			\item \textbf{Successfully Create Classes:} test to successfully create Instructor, and Student.
        		\end{itemize}
        		
        	\end{itemize}
        \end{itemize}
        
        \item \textbf{Persistence}
        \begin{itemize}
            \item \textbf{Reason to test:} This class is intrinsic to the job posting and application transaction. Without persistence capability the controllers have no data from which to derive the desired outcome
            \item \textbf{Test Cases:} 
            \begin{itemize}
                \item creation, modification and deletion of Course, Job, Application and Profile instances. 
            \end{itemize}
        \end{itemize}
        
    \end{itemize}
    
    \subsection{Classes to Not Test}
    \begin{itemize}
    	\item \textbf{Application Manager}
    	\begin{itemize}
    		\item \textbf{Reason to not test:} This class acts as a container for Application and Job. It does not perform any action; hence, there are no actions for it to test.
    	\end{itemize}
    \end{itemize}
    
    \begin{itemize}
    	\item \textbf{Profile Manager}
    	\begin{itemize}
    		\item \textbf{Reason to not test:} This class acts as a container for Student, Instructor, and Admin. It does not perform any action; hence, there are no actions for it to test.
    	\end{itemize}
    \end{itemize}
    
    \begin{itemize}
    	\item \textbf{Course Manager}
    	\begin{itemize}
    		\item \textbf{Reason to not test:} This class acts as a container for Course. It does not perform any action; hence, there are no actions for it to test.
    	\end{itemize}
    \end{itemize}
    
    \section{Techniques and Tools}
    
    The system is to be designed in a complete Test-Driven Development paradigm. Hence all test cases will be written before the actual controllers are implemented. Furthermore, the order in which the tests will be passed follows the same order as the requirements derived in the Requirements document of deliverable \#1. This will aid the classification and prioritization of each bi-weekly runs' objectives. In parallel to this method, the team will rely heavily on code revisions by the senior members of the team to ensure the code written is up to visual, performance and logical standards. In terms of frequency, individual nightly tests will be ran on all platforms of the system and bi-weekly builds and test will be done every Saturday to ensure the deliverable validates and verifies the given requirements. To facilitate the testing process, the team will use the following tools:
    
    \begin{itemize}
        \item \textbf{Unit Test Framework:} Automated, well documented and supported system to allow a standardized set of tests to be done regardless of the platform. JUnit and PHPUnit will be used.
        \item \textbf{Test Coverage Tool:} Ensure that the measure to which the tests and actions cover the source code is optimal by function, statement, branch and condition standards. Its outputs will be used during the code revision sessions. EclEmma and PHPUnit will be used for this purposes.
    \end{itemize}
    
    \section{Coverage Statistics}
    
    \textbf{TODOOOOOOO}
    To the extent possible, include a quantification of your testing goal (e.g., N\% statement coverage or M\% branch coverage). 
    
    Furthermore, discuss how often your tests are executed, i.e., what triggers your tests to be run? Finally, describe any differences between your desktop/laptop app, mobile app, and web app when it comes to unit testing.
    
\part{System Test Plan}
\section{Description}
The purpose of system testing is to validate the functionality of the system as a whole. System
tests are block-box tests, meaning they are done without any concern regarding the code - that is to
say, only the inputs and outputs are examined.\\\\
So as to examine the functionality of the system in an organized and systematic fashion, system
tests will be designed according to the various \textit{use cases} of the system. The use cases
provide concrete details concerning how each functionality of the system should behave, and in
particular how each functionality of the system should react to unexpected or faulty input.\\\\
The proposed system tests will be discussed below.
\subsection{Administrator-Specific Actions}
\begin{enumerate}
	\item \textbf{Sending Student Enrollment Data and Course List}
		\begin{enumerate}
			\item Valid data entered
			\begin{itemize}
				\item Enter appropriate data for student enrollment, as well as the course list, in the
					desktop application. Ensure that the proper XML files are generated.
			\end{itemize}
			\item Invalid student enrollment data entered
			\begin{itemize}
				\item Enter faulty student enrollment data (by omitting names, for example) and
					enter correct course list data in the desktop application. Ensure that an error
					message about invalid student data is made present. Moreover, assert that the
					corresponding student enrollment XML has not been generated at all, and that the
					course list XML file has been generated appropriately.
			\end{itemize}
			\item Invalid course list data entered
			\begin{itemize}
				\item Enter faulty course list data (by omitting required names, or by including
					non-integral course codes, for example) and enter correct student data in the
					desktop application. Ensure that an error message about invalid course list data
					is made present. Moreover, assert that the corresponding course list XML file
					has not been generated at all, and that the student enrollment XML file has been
					generated appropriately.
			\end{itemize}
		\end{enumerate}
	\item \textbf{Generating Initial Student Job Placements}
		\begin{enumerate}
			\item Valid XML Application file is present
				\begin{itemize}
					\item Ensure that the appropriate XML file containing data about student
						applications is present. Use desktop application to generate default TA
						placement.
						\begin{enumerate}
							\item Verify that an appropriate XML file containing data about the job
								offerings was generated.
							\item Verify that initial placement confines to the budget for each
								department.
							\item Verify that graduate students were selected with higher priority
								than undergraduate students.
							\item Verify that the same student was selected for as many jobs as
								possible for the same course.
						\end{enumerate}
				\end{itemize}
			\item XML Application file not found or is corrupted
				\begin{itemize}
					\item Place invalid XML file (containing syntax errors or type mismatches, for
						example) in the appropriate location, and attempt to generate default TA
						placement with the desktop application. Ensure that an error message about
						the invalidity of the XML file is output, and that no XML file about job
						offerings has been created.
				\end{itemize}
		\end{enumerate}
	\item \textbf{Accepting or Rejecting Instructor TA Modifications}
		\begin{enumerate}
			\item Accepting a modification
				\begin{itemize}
					\item Ensure that a list of all instructor modifications is present in the
						\textit{view proposed modifications} page. Select a modification for
						viewing, and select to accept the modification.
						\begin{enumerate}
							\item Ensure that the original allocations XML file has been overwritten
								using the data from the modification XML file that is currently
								active.
						\end{enumerate}
				\end{itemize}
			\item Rejecting a modification
				\begin{itemize}
					\item Ensure that a list of all instructor modifications is present in the
						\textit{view proposed modifications} page. Select a modification for
						viewing, and choose to reject the modification.
						\begin{enumerate}
							\item Ensure that the original allocations XML file has not been
								modified.
							\item Ensure that the modifications XML file has been deleted.
						\end{enumerate}
				\end{itemize}
		\end{enumerate}
	\item \textbf{Send Job Offers}
		\begin{enumerate}
			\item Modification XML files present
				\begin{itemize}
					\item Ensure that a TA placement modification XML file is present. Attempt
						sending job offers in the \textit{view allocation} page.
						\begin{enumerate}
							\item Ensure that no \texttt{offers.xml} file is generated.
							\item Ensure that an error message informing the user that there are
								still outstanding modifications is output.
						\end{enumerate}
				\end{itemize}
			\item No modification XML files are present
				\begin{itemize}
					\item Remove all modification XML files, and ensure that initial allocation XML
						file is present. Attempt sending job offers in the \textit{view allocation}
						page.
						\begin{enumerate}
							\item Ensure that the \texttt{offers.xml} file has been generated with
								appropriate job offer data corresponding to the TA allocation.
						\end{enumerate}
				\end{itemize}
		\end{enumerate}
\end{enumerate}
\subsection{Instructor Use-Cases}
\begin{enumerate}
	\item \textbf{Publish Job Posting}
		\begin{enumerate}
			\item No course data found
				\begin{itemize}
					\item Remove all course data, or delete the \texttt{courses.xml} file. Attempt
						to open the page containing the \textit{publish job posting} form. Ensure
						that the page does not open, and an error message informing the user that no
						courses are available is output.
				\end{itemize}
			\item Course and Instructor data present
				\begin{itemize}
					\item Ensure the presence of the \texttt{courses.xml} and \texttt{profiles.xml}
						files, and make sure at least one course is present. Then, open the
						\textit{publish job posting} form and assert that for the indicated
						Instructor, all of its courses, and only courses that are being taught by
						that given professor, are available for selection in the Course list.
				\end{itemize}
			\item Valid input entered, no previous XML file found
				\begin{itemize}
					\item Specify valid information corresponding to a job posting in the
						\textit{publish job posting} form, and submit the data. Ensure that a XML
						file called \texttt{applications.xml} was generated and contains correct
						data concerning the job postings that were submitted.
				\end{itemize}
			\item Valid input entered, with previous XML file in place
				\begin{itemize}
					\item With a previously-generated \texttt{applications.xml} in place, attempt
						submitting a job posting using the \textit{publish job posting} form. Ensure
						that the XML file has been modified such that the new job postings have been
						appended after the previous ones (therefore, ensure no data was lost).
				\end{itemize}
			\item Type mismatches or empty fields submitted
				\begin{itemize}
					\item Attempt submitting a job posting, leaving fields blank or including type
						mismatches, such as a non-integral input in the Salary field. Ensure that
						the \texttt{applications.xml} file was not modified (and if it did not
						previously exist, ensure that it has not been generated).
				\end{itemize}
			\item Valid, but unsound data submitted
				\begin{itemize}
					\item Attempt submitting a job posting with unsound data, such as a start time
						that is later than an end time.
						\begin{enumerate}
							\item Ensure that the appropriate error message is output.
							\item Ensure that the \texttt{applications.xml} file was not modified
								(and if it did not previously exist, ensure that it has not been
								generated).
						\end{enumerate}
				\end{itemize}
		\end{enumerate}
	\item \textbf{Modify initial TA allocation}
		\begin{enumerate}
			\item No initial allocation data found
				\begin{itemize}
					\item Remove XML file with initial allocation data. Attempt to open TA placement
						modification page. Ensure that the page does not open, and that an error
						message informing the user that no allocation has been made is output.
				\end{itemize}
			\item Valid modifications entered
				\begin{itemize}
					\item Enter valid modifications in the modification form, and submit.
						\begin{enumerate}
							\item Ensure that the initial allocation XML file has not been
								modified.
							\item Ensure that a seperate XML file containing the data about the
								modifications has been created appropriately.
						\end{enumerate}
				\end{itemize}
			\item Invalid modifications entered
				\begin{itemize}
					\item Enter modifications that have empty fields or type mismatches, for
						example. Attempt submitting modifications.
						\begin{enumerate}
							\item Verify that an error message containing the details of the
								invalidity of the input has been shown.
							\item Ensure that the initial allocation XML file has not been modified.
							\item Ensure that no seperate XML files containing modification data
								have been generated.
						\end{enumerate}
				\end{itemize}
		\end{enumerate}
	\item \textbf{Evaluate TA/Grader}
		\begin{enumerate}
			\item No student XML data found or no TA's associated to current professor.
				\begin{itemize}
					\item Remove all student data, or remove job data for a specific professor.
						Under such professor's account, attempt to open the \textit{evaluate TA}
						page. Ensure that the page does not open, and that a message informing the
						user that no TA's are available is output.
				\end{itemize}
			\item TA data available
				\begin{itemize}
					\item Open the \textit{evaluate TA} page.
						\begin{enumerate}
							\item Ensure that only the TA's that worked for courses taught by the
								selected professor are available for evaluation.
							\item After submission, ensure that the Student that was evaluated had
								its \texttt{experience} field modified in the \texttt{profiles.xml}
								file.
						\end{enumerate}
				\end{itemize}
		\end{enumerate}
\end{enumerate}
\subsection{Student Use-Cases}
\begin{enumerate}
	\item \textbf{Apply to Job Posting}
		\begin{enumerate}
			\item No Job data available
			\begin{itemize}
				\item Remove \texttt{applications.xml} file, or ensure that it contains no Job data.
					Attempt opening the \textit{apply to job posting} page. Ensure that the page
					does not open and that a message informing the user that Jobs have not been
					created yet is output.
			\end{itemize}
			\item Valid input entered
			\begin{itemize}
				\item Ensure that Job data is available. Open \textit{apply to job posting} page and
					select a Job to apply to. Ensure that all Jobs that were present in the XML data
					are available in the selection list. Ensure that after submission, the
					\texttt{applications.xml} file was modified appropriately, and that no data has
					been lost.
			\end{itemize}
		\item Invalid input submitted
			\begin{itemize}
				\item Ensure that Job data is available. Open the \textit{apply to job posting} page
					and select a Job to apply to. Leave Job field blank, and attempt submitting.\
					\begin{enumerate}
						\item Ensure that an error message informing the user that no Job was
							selected has been output.
						\item Ensure that the \texttt{applications.xml} file has not been modified.
					\end{enumerate}
			\end{itemize}
		\end{enumerate}
	\item \textbf{Submit Personal Details}
		\begin{enumerate}
			\item Open \textit{modify profile} page. Ensure that all current data for the current
				Student is present in the form. Modify data in the forms and submit. Ensure that the
				\textit{profiles.xml} file has been modified appropriately.
		\end{enumerate}
	\item \textbf{Accept or Reject Job Offerings}
		\begin{enumerate}
			\item Open the \textit{job offers} page. Ensure that only jobs offered to the current
				Student are listed. Attempt accepting a job offer and rejecting a job offer. Ensure
				that the \texttt{offers.xml} file has been modified accordingly.
		\end{enumerate}
\end{enumerate}
\end{document}
