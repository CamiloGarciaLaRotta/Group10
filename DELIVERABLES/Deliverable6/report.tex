\documentclass[12pt]{article}

\usepackage{amsmath,amssymb,amsfonts} %In case we need math stuff
\usepackage{graphicx} %For inserting images and stuff
\usepackage{listings} %For inserting code snippets
\usepackage{enumerate} %For fancy enumeration
\usepackage[margin=1.3cm]{geometry} %Nice margin setting

\title{ECSE 321 - Intro to Software Engineering\\Deliverable 6 Report}
\author{Harley Wiltzer\\Camilo Garcia La Rotta\\Jake Shnaidman\\Robert Attard\\Matthew Lesko}
\date

\begin{document}
\pagenumbering{gobble} %No page number on title page
\maketitle
\newpage
\pagenumbering{arabic} %Arabic numeral page numbers on regular pages

\newpage
\textbf{1. Implementation}


\textbf{All the requirements formally identified by an ID are references to the Deliverable 1.}
\begin{itemize}

    \item \textbf{Functional:} Total implementation of the R. 1.1.1.1 to 1.1.1.4 with the caveat of serialization to text file instead of an xml format for the Web platform. This was a design choice made during the 2nd deliverable to use the same persistence methodology seen in Event Registration. R. 1.1.1.5 was not accomplished, a database was voted as too much of a frame for the access and retrieval of data. A much lightweight approach of storage in text files was preferred. R 1.1.1.6 is partially accomplished. Mobile and Desktop are fully capable of using each others xml outputs. Web remained incompatible due to its plain text output serialization.
        
    \item \textbf{Non-Functional:} From section R. 1.2.1 only R 1.2.1.3 will ship with the final report. All other requirements in this section were dropped from development as they related to an extra scheduling algorithm design the team lacked time to develop
    from section R. 1.2.2 all platforms provide a secure authentication procedure linked to a profile. The only requirement not achieved was 1.2.2.6 as NTRU cryptosystems required more research on our part to be able to implement from section 1.2.3 all requirements were achieved, providing a reliable cross platform version of the product was one of the main goals of the team from section 1.2.4 all applications work under the same colour pallet, but as a caveat Mobile is the only platform without light/dark theme mode. It was chosen to rely on androids native accessibility options to change color gradients on the cellphone.
        

	\item \textbf{Desktop:} From section R 1.1.2 all but 1.1.2.9 was accomplished. As demanded by the client, the Desktop application is fully able to perform the tasks of an Instructor, Department and Student Profile which can be found in more depth in the Mobile and Web section of this report.
	
	\item \textbf{Web:} The web service is capable of dual profile actions. An administrator can log in and create courses instances and instructor profiles as well as linking them. An instructor can log in and publish job postings and manage the received applications as well as evaluating TAs.
    
	\item \textbf{Mobile:} The shipped application provides the whole capability of appying for job, Accepting/Rejecting an offer
 Creating and Modifying the users student profile 
	   % \item 1.4.1 - The mobile application only allows student profiles to login
	   % \item 1.4.2 - The mobile application was written in java
	   % \item 1.4.3 - The mobile application only used libraries provided by android studio and no other tools were used
	   % \item 1.4.4 - The students may store their student ID in their description as a string.
	   % \item 1.4.5 - The application uses a controller to load job postings from persistence
	   % \item 1.4.6 - The application lists all the job postings to the user in a separate view
	   % \item 1.4.7 - This was not fulfilled since the administration can view other jobs that the student has applied for and can control whether or not he will get the job
	   % \item 1.4.8 - The user is not able to specify his preferences and rank his applications. He instead is given the choice to choose which offers he will ultimately accept
	   % \item 1.4.9 - The mobile application retrieves job offers from persistence
	   % \item 1.4.10 - The mobile application displays job offers to the user
	   % \item 1.4.11 - The student may accept or reject job offers

\end{itemize}

\textbf{2. Usability of Application}
%% -- The final state of the application, do all three platforms work as intended? Which Use Cases are each of the three applications capable of performing? Should the application be run in a certain way?

\begin{itemize}
	\item \textbf{Desktop:} The desktop program shall be run by double clicking on the .jar file of the program. The output directory gets initialized in the same directory as the .jar file is located in. All saved data (persistence data) is stored in this output folder. No installation is necessary with the use of the .jar file. Full deletion requires the deletion of the output folder and the .jar file, and if the user has forked the repository, the user would need to delete the local repository as well.
	
	\item \textbf{Web:} The Web application is accessed through the web and doesn't host any permanent or temporary files client-side. The user is required to access the server that hosts the website.
	
	\item \textbf{Mobile:} The mobile program shall be released as an apk file that may be installed on an Android phone. Once the apk file is installed, the user may use the application on the phone. 
\end{itemize}

\textbf{3. Testing of Applications}


%% -- Talk about the final state of testing. Have all unit tests been done -> Are we confident enough that certain inputs will not cause problems? Do they all pass? Coverage?
%% -- Is all integration testing done -> Are we confident enough with certain feaures working together? Are there any concerns for certain features? Will there be more testing in the future?
%% -- Is all system testing done -> Are we confident enough to release the product?
\begin{itemize}
	\item \textbf{Unit Testing:}
	\item \textbf{Integration Testing:}
	\item \textbf{System Testing:}
\end{itemize}

\textbf{4. Release Pipeline}


%% -- Ready for release or not?
%% -- Where and how shall it be released? Shall we release it as intended in the release plan?
The release pipeline shall follow the plan as in deliverable 4's report. Every release shall follow semantic versioning: MAJOR.MINOR.PATCH. With the first release being V1.0.0. During deployment, the Travis script shall invoke the scripts: ant export-desktop, ant export-web, and ant export-mobile. These shall export the necessary jar, apk, and web files.
\begin{itemize}
	\item \textbf{Desktop:} The Executable Desktop JAR archive shall be released. The APP/Desktop directory shall be released as a repository.
	\item \textbf{Web:} The Web Application zip archive shall be released. The APP/Web directory shall be released as a repository.
	\item \textbf{Mobile:} The apk file of the android program shall be released. The APP/Mobile directory shall be released as a repository.
\end{itemize}

It is interesting to note that a good part of the release pipeline time budget was allocated for the build script. Apart from the base functionalities it is also capable of packaging the web app, start the xampp server and open a browser with the address of the web service.Install the mobile app automatically to a connected android device

\textbf{5. Responsibilities for Each Phase}
\begin{itemize}
    \item \textbf{Harley Wiltzer: }
    \item \textbf{Camilo Garcia: }
    \item \textbf{Jacob Shnaidman:(420+h) }
    \item \textbf{Robert Attard: }
    \item \textbf{Matthew Lesko: (40+h)} 
    % Involved in Design of Domain Model, Software Architecture, Use Case Diagram, and Behavioral Sequence Diagram. Updating necessary changes to Domain Model and Detailed Domain Model during this phase. Unit Testing Desktop application. Involved in implementing new Desktop application features under a Test Driven Development Paradigm. Involved in Deployment Phase of Software System. Suggested to use Travis CI as Deployment Tool and GitHub Releases as a release hosting solution. Implemented Travis CI as a CI tool.
\end{itemize}

\end{document}