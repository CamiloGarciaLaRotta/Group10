\documentclass[12pt]{report}

\usepackage{amsmath,amssymb,amsfonts} %In case we need math stuff
\usepackage{graphicx} %For inserting images and stuff
\usepackage{listings} %For inserting code snippets
\usepackage{enumerate} %For fancy enumeration
\usepackage{hyperref}
\usepackage{pdfpages}
\usepackage{float}
\usepackage[margin=2cm]{geometry} %Nice margin setting

\renewcommand*\thesection{\arabic{section}}

\title{ECSE 321 - Intro to Software Engineering\\Design Specification Document - Deliverable 3}
\author{Harley Wiltzer\\Camilo Garcia La Rotta\\Jake Shnaidman\\Robert Attard\\Matthew Lesko}
\date{March 19, 2017}

\begin{document}
\pagenumbering{gobble} %No page number on title page
\maketitle
\newpage
\pagenumbering{arabic} %Arabic numeral page numbers on regular pages
\tableofcontents
\part{Unit Tests}
\section{Course Unit Test}
	\textbf{METHOD UNDER TEST: Constructor} \newline
	\text{*Note:} For cases in which a message, stating for a wrong input, is output from the system, then the changes/creations/deletions of objects should not persist.\newline
	\textbf{Test Cases For: className Inputs}
	\begin{flushleft}
		\begin{tabular}{ | l | l | l | l | }
			\hline
			Test Cases for Constructor & className Input & cdn Input & TimeBudget Input \\ \hline
			Test Case no.1 & null & 101 & 10.00 \\ \hline
			Test Case no.2 & "" & 101 & 10.00  \\ \hline
			Test Case no.3 & "1234567890" & 101 & 10.00  \\ \hline
			Test Case no.4 & " " & 101 & 10.00 \\ \hline
		\end{tabular}
	\end{flushleft}

	\begin{flushleft}
		\begin{tabular}{ | l | l | }
			\hline
			Test Cases for Constructor & Expected Output \\ \hline
			Test Case no.1 & Course name cannot be empty! \\ \hline
			Test Case no.2 & Course name cannot be empty! \\ \hline
			Test Case no.3 &  (className gets saved in persistence layer) \\ \hline
			Test Case no.4 & Course name cannot be empty! \\ \hline
			
		\end{tabular}
	\end{flushleft}

	\textbf{Test Cases For: cdn Inputs}
	\begin{flushleft}
		\begin{tabular}{ | l | l | l | l | }
			\hline
			Test Cases for Constructor & className Input & cdn Input & TimeBudget Input \\ \hline
			Test Case no.5 & ECSE & 101 and 101 & 10.00 \\ \hline
			Test Case no.6 & ECSE & null & 10.00 \\ \hline
			Test Case no.7 & ECSE & -1 & 10.00 \\ \hline
			Test Case no.8 & ECSE & "one hundred and one" & 10.00 \\ \hline
			Test Case no.9 & ECSE & " " & 10.00 \\ \hline
		\end{tabular}
	\end{flushleft}
	
	\begin{flushleft}
		\begin{tabular}{ | l | l | }
			\hline
			Test Cases for Constructor & Expected Output \\ \hline
			Test Case no.5 & Cannot input Non-Unique CDN! \\ \hline
			Test Case no.6 & Cannot input empty CDN! \\ \hline
			Test Case no.7 & CDN must be positive! \\ \hline
			Test Case no.8 & Cannot input alphabetical characters for CDN!  \\ \hline
			Test Case no.9 & Cannot input empty CDN!  \\ \hline			
		\end{tabular}
	\end{flushleft}

	\textbf{Test Cases For: graderTimeBudget/taTimeBudget Inputs} 
	\begin{flushleft}
		\begin{tabular}{ | l | l | l | l | }
			\hline
			Test Cases for Constructor & className Input & cdn Input & TimeBudget Input \\ \hline
			Test Case no.10 & ECSE & 101 & null \\ \hline
			Test Case no.11 & ECSE & 101 & "" \\ \hline
			Test Case no.12 & ECSE & 101 & "ten" \\ \hline
			Test Case no.13 & ECSE & 101 & -10.00 \\ \hline
		\end{tabular}
	\end{flushleft}
	
	\begin{flushleft}
		\begin{tabular}{ | l | l | }
			\hline
			Test Cases for Constructor & Expected Output \\ \hline
			Test Case no.10 & Cannot input empty TimeBudget! \\ \hline
			Test Case no.11 & Cannot input empty TimeBudget! \\ \hline
			Test Case no.12 & Cannot input alphabetical characters for TimeBudget!  \\ \hline
			Test Case no.13 & Grader Budget must be positive! TA Budget must be positive!    \\ \hline
		\end{tabular}
	\end{flushleft}
	
	\newpage
	\textbf{METHOD UNDER TEST: addJob} \newline
	\textbf{Test Cases For: addJob Inputs} 
	\begin{flushleft}
		\begin{tabular}{ | l | l | l | l | l | l | l | }
			\hline
			Test Cases & startTime & endTime & Day & Salary & Requirements & Instructor \\ \hline
			Test Case no.14 & null & null & null & null & null & null \\ \hline
			Test Case no.15 & 10:00 & 15:00 & Monday & 15.00 & Bachelors & Default Instructor \\ \hline
		\end{tabular}
	\end{flushleft}
	
	\begin{flushleft}
		\begin{tabular}{ | l | l | }
			\hline
			Test Cases & Expected Output \\ \hline
			Test Case no.14 & Invalid Input for Job! \\ \hline
			Test Case no.15 & (Job gets saved in persistence layer) \\ \hline
		\end{tabular}
	\end{flushleft}

	\textbf{METHOD UNDER TEST: addJobAt} \newline
	\textbf{Test Cases For: addJobAt Inputs}
	\begin{flushleft}
		\begin{tabular}{ | l | l | l | l | l | }
			\hline
			Test Cases & Job1 Input & Index1 Input & Job2 Input & Index2 Input \\ \hline
			Test Case no.16 & Job1 & null & Job2 & null \\ \hline
			Test Case no.17 & Job1 & 1 & Job2 & 1 \\ \hline
		\end{tabular}
	\end{flushleft}
	
	\begin{flushleft}
		\begin{tabular}{ | l | l | }
			\hline
			Test Cases & Expected Output \\ \hline
			Test Case no.16 & Cannot add Job at null index! \\ \hline
			Test Case no.17 & Cannot add Job at occupied index! \\ \hline
		\end{tabular}
	\end{flushleft}

	\textbf{METHOD UNDER TEST: addOrMoveJobAt} \\
	\textbf{Test Cases For: addOrMoveJobAt Inputs}
	\begin{flushleft}
		\begin{tabular}{ | l | l | l | l | }
			\hline
			Test Cases & Job Input & Index Input & New Index Input \\ \hline
			Test Case no.18 & Job & 1 & 2 \\ \hline
			Test Case no.19 & Job & 1 & 0 \\ \hline
		\end{tabular}
	\end{flushleft}
	
	\begin{flushleft}
		\begin{tabular}{ | l | l | }
			\hline
			Test Cases & Expected Output \\ \hline
			Test Case no.18 & (Job gets moved to a new index successfully) \\ \hline
			Test Case no.19 & (Job gets moved to a new index successfully) \\ \hline
		\end{tabular}
	\end{flushleft}
\newpage
\section{Job Unit Test}
	\textbf{*Note:} Since the Classes Tutorial and Laboratory inherit from the Job class; the test cases for these two classes will be identical to the Job Class. The inputs aCourse and aInstructor are simply valid Course and Instructor inputs. \\
	\textbf{METHOD UNDER TEST: Constructor} \newline
	\textbf{Test Cases For: startTime and endTime Inputs}
	\begin{flushleft}
		\begin{tabular}{ | l | l | l | l | l | l | l | l | }
			\hline
			Test Cases & startTime & endTime & Day & Salary & Requirements & Course & Instructor \\ \hline
			Test Case no.1 & null & null & Monday & 10.00 & Bachelors & aCourse & aInstructor \\ \hline
			Test Case no.2 & "" & "" & Monday & 10.00 & Bachelors & aCourse & aInstructor \\ \hline
			Test Case no.3 & 10:00 & 9:00 & Monday & 10.00 & Bachelors & aCourse & aInstructor \\ \hline
			Test Case no.4 & 21:00 & 22:00 & Monday & 10.00 & Bachelors & aCourse & aInstructor \\ \hline
		\end{tabular}
	\end{flushleft}
	
	\begin{flushleft}
		\begin{tabular}{ | l | l | }
			\hline
			Test Cases & Expected Output \\ \hline
			Test Case no.1 & Input Time cannot be empty! \\ \hline
			Test Case no.2 & Input Time cannot be empty! \\ \hline
			Test Case no.3 & startTime cannot be before endTime! \\ \hline
			Test Case no.4 & Times cannot be outside working hours! \\ \hline
		\end{tabular}
	\end{flushleft}

	\textbf{Test Cases For: Requirements Inputs}
	\begin{flushleft}
		\begin{tabular}{ | l | l | l | l | l | l | l | l | }
			\hline
			Test Cases & startTime & endTime & Day & Salary & Requirements & Course & Instructor \\ \hline
			Test Case no.5 & 10:00 & 12:00 & Monday & 10.00 & null & aCourse & aInstructor \\ \hline
			Test Case no.6 & 10:00 & 12:00 & Monday & 10.00 & "" & aCourse & aInstructor \\ \hline
			Test Case no.7 & 10:00 & 12:00 & Monday & 10.00 & " " & aCoursee & aInstructor \\ \hline
		\end{tabular}
	\end{flushleft}
	
	\begin{flushleft}
		\begin{tabular}{ | l | l | }
			\hline
			Test Cases & Expected Output \\ \hline
			Test Case no.5 & Requirements cannot be empty! \\ \hline
			Test Case no.6 & Requirements cannot be empty! \\ \hline
			Test Case no.7 & Requirements cannot be empty! \\ \hline
		\end{tabular}
	\end{flushleft}

	\textbf{Test Cases For: Course and Instructor Inputs}
	\begin{flushleft}
		\begin{tabular}{ | l | l | l | l | l | l | l | l | }
			\hline
			Test Cases & startTime & endTime & Day & Salary & Requirements & Course & Instructor \\ \hline
			Test Case no.8 & 10:00 & 12:00 & Monday & 10.00 & PHD Student & null & aInstructor \\ \hline
			Test Case no.9 & 10:00 & 12:00 & Monday & 10.00 & PHD Student & aCourse & null \\ \hline
		\end{tabular}
	\end{flushleft}
	
	\begin{flushleft}
		\begin{tabular}{ | l | l | }
			\hline
			Test Cases & Expected Output \\ \hline
			Test Case no.8 & Invalid Course Input! \\ \hline
			Test Case no.9 & Invalid Instructor Input! \\ \hline
		\end{tabular}
	\end{flushleft}

	\newpage
	\textbf{Test Cases For: Salary Inputs}
	\begin{flushleft}
		\begin{tabular}{ | l | l | l | l | l | l | l | l | }
			\hline
			Test Cases & startTime & endTime & Day & Salary & Requirements & Course & Instructor \\ \hline
			Test Case no.10 & 10:00 & 12:00 & Monday & null & PHD Student & aCourse & aInstructor \\ \hline
			Test Case no.11 & 10:00 & 12:00 & Monday & " " & PHD Student & aCourse & aInstructor \\ \hline
			Test Case no.12 & 10:00 & 12:00 & Monday & "ten10" & PHD Student & aCourse & aInstructor \\ \hline
			Test Case no.13 & 10:00 & 12:00 & Monday & -10 & PHD Student & aCourse & aInstructor \\ \hline
		\end{tabular}
	\end{flushleft}
	
	\begin{flushleft}
		\begin{tabular}{ | l | l | }
			\hline
			Test Cases & Expected Output \\ \hline
			Test Case no.10 & Salary cannot be empty! \\ \hline
			Test Case no.11 & Salary cannot be empty! \\ \hline
			Test Case no.12 & Salary cannot have alphabetical characters! \\ \hline
			Test Case no.13 & Salary cannot be negative! \\ \hline
		\end{tabular}
	\end{flushleft}

	\textbf{Test Cases For: Day Inputs}
	\begin{flushleft}
		\begin{tabular}{ | l | l | l | l | l | l | l | l | }
			\hline
			Test Cases & startTime & endTime & Day & Salary & Requirements & Course & Instructor \\ \hline
			Test Case no.14 & 10:00 & 12:00 & null & 10.00 & PHD Student & aCourse & aInstructor \\ \hline
			Test Case no.15 & 10:00 & 12:00 & Saturday & 10.00 & PHD Student & aCourse & aInstructor \\ \hline
		\end{tabular}
	\end{flushleft}
	
	\begin{flushleft}
		\begin{tabular}{ | l | l | }
			\hline
			Test Cases & Expected Output \\ \hline
			Test Case no.14 & Date cannot be empty! \\ \hline
			Test Case no.15 & Date cannot be a weekend day! \\ \hline
		\end{tabular}
	\end{flushleft}

\newpage
\section{Profile}

\part{Integration Tests}
The integration tests are split into two parts. One part is the Java which tests the functionality of the controllers that operate in the Desktop and Mobile application in order to test the integration of multiple entity classes. The other is the PHP which tests how numerous classes can be integrated together. 

\section{Java}
All together, the Java code has tested that all classes function harmoniously as they are needed to within our Mobile and Desktop application.

\subsection{Profile Controller}
This controller tests the integration of the profile class with the courses class as well as persistence and profile manager class. This test achieved 100\% code coverage using ECLEmma. \\\\
It tested for null and empty input in all fields within the controller and ensured that it was handled. It also ensured that profiles that had incorrect input were not added to the system (were not persisted in the XML)

\subsection{Course Controller}
This controller tests the integration of the course class with the course controller, course manager and persistence. This test achieved 99.2\% code coverage. \\\\
It tested for null and empty input in all String fields in the controller and ensured that it was handled. It also checked to see that CDNs, and time budgets could not be negative. Most importantly, the integration test ensured that exceptional input did not propagate into the persistence layer.

\subsection{Application Controller}
This controller tests the integration of the application class with the application controller, application manager, profile manager, profile controller, persistence, course class, profile class, and job class. This test achieved 100\% code coverage \\\\
It tested that null and empty input in all String fields in the controller as well as negative salaries were handled by exceptions. It also tested that these errors did not propagate into persistence. 

\section{PHP}
\subsection{Job Integration}
	Given that the web client is inteded to be used by an instructor, the integration testing relates the job class to all the other classes in various scenarios and use-cases by utilising their controllers. While the unit test for profile and course used some implementation of their respective controllers, here we try to use only the controllers as much as possible for all interactions with the job class and the application controller, which handles the operations needed for the job class to be created and used.

\subsection{Test Cases}
	The integration testing for the application controller and job class run through various input cases. These include creating a valid job with the controller, creating a job with no linked course, creating a job with an invalid course CDN, creating a job with no instructor, and the various combinations of remaining inputs for start and end time, salary and position.\\

	As of yet integration testing for the web client is not complete and no coverage metrics can be generated. Ideally, after testing all possible use cases that apply to the web client (the model generated code that may not be used), coverage should be around 90\%.\\
\part{System Tests}
System tests have been carried out incrementally as new use cases for the system have been added.
Since the desktop app is the most mature of the three platforms in development, it has experienced
by far the most system testing, as it currently implements approximately twice as many use cases as
the other platforms combined. Therefore, the system tests for the desktop app will be described
below, and will be structured by the use cases that they examine.

\section{The Register Profile Use Case}
\label{s:six}
\textbf{Goal: To ensure that profiles are created properly using the RegisterView}
\subsection*{Process}
This use case was tested by entering various inputs to the RegisterView class, observing the output
messages on the view, and verifying the persisted XML upon completion. Inputs were decided according
to equivalence partitioning, so an input was tested for each class of error that can occur from user
input, and of course a valid input was tested as well.
\subsection*{Results}
All tests were passed successfully. Invalid input test cases all showed proper error messages on the
view, and proper input test cases showed correct confirmation messages. Furthermore, it was noted
that XML was only generated when proper input was submitted.
\subsection*{Conclusion}
The RegisterView UI functions correctly, and reliably registers profiles to the system with proper
persistence.

\section{The Upload Course Data Use Case}
\textbf{Goal: To ensure that courses are created properly using the CourseView}
\subsection*{Process}
The process followed a very similar strategy to that of the system test described in
\hyperref[s:six]{section 6}. Equivalence classes were chosen as test cases, and XML output as well
as messages on the UI were verified.
\subsection*{Results}
All tests were passed successfully. XML was created only when valid input was passed in, and all UI
messages reflected the validity and errors of the input.
\subsection*{Conclusion}
The CourseView UI functions correctly, and reliably uploads course data to the system with proper
persistence.

\section{The Publish Job Posting Use Case}
\textbf{Goal: To ensure that instructors may publish jobs for the courses that they teach, and to
verify that only sane job postings are processed.}
\subsection*{Process}
This is the first system test that depends on the functioning of other use cases. Given the results
of the previous system tests, there was confidence that the behavior of the current use case will
not be hindered by its dependencies. It was first verified that the instructor may only choose to
select courses that the profile is said to teach for the job posting. Then, it was verified that
upon selecting a different professor, the list of courses to choose from updates (with a correct
list of courses) immediately. Furthermore, aside from plain empty text fields, the PublishJobView
has several input fields that can be erroneous due to incorrect input types, and even correct input
types that represent non-sensical data. For example, the time budget fields must be integral types,
and the start time of the job must occur before the end time. Therefore, configurations where the
end time occurs before the start time were tested, and configurations where time budgets with
non-integral time budgets were tested, along with the usual tests of empty fields and valid input.
Furthermore, it was verified that XML was only generated upon \textit{strictly valid} input, that is
to say, input that has correctly-typed and non-empty fields \textit{and} only logically-sane inputs.
Finally, the application was closed and restarted. Then, the ApplicationView was opened to ensure that
the job postings were present in the list,
which would confirm that the persistence is functioning in such a way that its stored data may be
retrieved Finally, the ApplicationView was opened to ensure that the job postings were present in
the list, which would confirm that the persistence is functioning in such a way that its stored data
may be retrieved.
\subsection*{Results}
Unfortunately, not all test were passed immediately. It was noticed that it was possible to submit
job postings with start times that occur after end times, which is undesirable. After many tests,
the testers realized that by \textit{nudging} the start time field, or simply changing its value and
returning it to its initial value, the error vanished. Thus, the PublishJobView was modified to
explicitly set the values of the start time and end time fields whenever the view refreshes. This
ultimately solved the error. Otherwise, all other functionalities were successful, and XML was only
being generated when the system interpreted that the input data was valid.
\subsection*{Conclusion}
Although there was a slight failure early in testing, it appears to have been resolved, and
currently the Publish Job Posting Use Case is considered to be functional. This use case has been
tested extensively since its failures, as it is a dependency for several other use cases. Since the
error was resolved, no other errors have been noticed.

\section{The Apply to Job Posting Use Case}
\textbf{Goal: To ensure that students may apply to the job postings that were posted.}
\subsection*{Process}
This use case was very simple to test, since there are very few ways in which input can be invalid.
In fact, the only inputs are the selections from two lists, which may be null. Therefore, there
were four test cases, including one where the student list had no selection, one where the job list
had no selection, one where neither list had a selection, and of course one where both lists had a
selected value. It was ensured that proper error or confirmation messages were excreted upon each
submission of input, and of course XML persistence was verified. In the process, the Publish Job
Posting Use Case was also examined, as upon selecting jobs from the list the job's data appears in a
text area, so job data was observed as a testament to the persistence in PublishJobView.
\subsection*{Results}
All tests were passed. Appropriate error messages were output for all erroneous input
configurations, and a correct confirmation message was output for the valid input class.
Furthermore, persistence was only generated when valid input was submitted. Moreover, it was noticed
that by observing the job data in the job description text area, all job data appears to be correct,
further emphasizing the validity of the Publish Job Posting Use Case implementation.
\end{document}
