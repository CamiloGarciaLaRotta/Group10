\documentclass[12pt]{report}

\usepackage{amsmath,amssymb,amsfonts} %In case we need math stuff
\usepackage{graphicx} %For inserting images and stuff
\usepackage{listings} %For inserting code snippets
\usepackage{enumerate} %For fancy enumeration
\usepackage{hyperref}
\usepackage{pdfpages}
\usepackage{float}
\usepackage[margin=2cm]{geometry} %Nice margin setting

\renewcommand*\thesection{\arabic{section}}

\title{ECSE 321 - Intro to Software Engineering\\Design Specification Document - Deliverable 3}
\author{Harley Wiltzer\\Camilo Garcia La Rotta\\Jake Shnaidman\\Robert Attard\\Matthew Lesko}
\date{March 19, 2017}

\begin{document}
\pagenumbering{gobble} %No page number on title page
\maketitle
\newpage
\pagenumbering{arabic} %Arabic numeral page numbers on regular pages
\tableofcontents
\part{Unit Tests}
\section{Course Unit Test}
	\textbf{METHOD UNDER TEST: Constructor} \newline
	\text{*Note:} For cases in which a message, stating for a wrong input, is output from the system, then the changes/creations/deletions of objects should not persist.\newline
	\textbf{Test Cases For: className Inputs}
	\begin{flushleft}
		\begin{tabular}{ | l | l | l | l | }
			\hline
			Test Cases for Constructor & className Input & cdn Input & TimeBudget Input \\ \hline
			Test Case no.1 & null & 101 & 10.00 \\ \hline
			Test Case no.2 & "" & 101 & 10.00  \\ \hline
			Test Case no.3 & "1234567890" & 101 & 10.00  \\ \hline
			Test Case no.4 & " " & 101 & 10.00 \\ \hline
		\end{tabular}
	\end{flushleft}

	\begin{flushleft}
		\begin{tabular}{ | l | l | }
			\hline
			Test Cases for Constructor & Expected Output \\ \hline
			Test Case no.1 & Course name cannot be empty! \\ \hline
			Test Case no.2 & Course name cannot be empty! \\ \hline
			Test Case no.3 &  (className gets saved in persistence layer) \\ \hline
			Test Case no.4 & Course name cannot be empty! \\ \hline
			
		\end{tabular}
	\end{flushleft}

	\textbf{Test Cases For: cdn Inputs}
	\begin{flushleft}
		\begin{tabular}{ | l | l | l | l | }
			\hline
			Test Cases for Constructor & className Input & cdn Input & TimeBudget Input \\ \hline
			Test Case no.5 & ECSE & 101 and 101 & 10.00 \\ \hline
			Test Case no.6 & ECSE & null & 10.00 \\ \hline
			Test Case no.7 & ECSE & -1 & 10.00 \\ \hline
			Test Case no.8 & ECSE & "one hundred and one" & 10.00 \\ \hline
			Test Case no.9 & ECSE & " " & 10.00 \\ \hline
		\end{tabular}
	\end{flushleft}
	
	\begin{flushleft}
		\begin{tabular}{ | l | l | }
			\hline
			Test Cases for Constructor & Expected Output \\ \hline
			Test Case no.5 & Cannot input Non-Unique CDN! \\ \hline
			Test Case no.6 & Cannot input empty CDN! \\ \hline
			Test Case no.7 & CDN must be positive! \\ \hline
			Test Case no.8 & Cannot input alphabetical characters for CDN!  \\ \hline
			Test Case no.9 & Cannot input empty CDN!  \\ \hline			
		\end{tabular}
	\end{flushleft}

	\textbf{Test Cases For: graderTimeBudget/taTimeBudget Inputs} 
	\begin{flushleft}
		\begin{tabular}{ | l | l | l | l | }
			\hline
			Test Cases for Constructor & className Input & cdn Input & TimeBudget Input \\ \hline
			Test Case no.10 & ECSE & 101 & null \\ \hline
			Test Case no.11 & ECSE & 101 & "" \\ \hline
			Test Case no.12 & ECSE & 101 & "ten" \\ \hline
			Test Case no.13 & ECSE & 101 & -10.00 \\ \hline
		\end{tabular}
	\end{flushleft}
	
	\begin{flushleft}
		\begin{tabular}{ | l | l | }
			\hline
			Test Cases for Constructor & Expected Output \\ \hline
			Test Case no.10 & Cannot input empty TimeBudget! \\ \hline
			Test Case no.11 & Cannot input empty TimeBudget! \\ \hline
			Test Case no.12 & Cannot input alphabetical characters for TimeBudget!  \\ \hline
			Test Case no.13 & Grader Budget must be positive! TA Budget must be positive!    \\ \hline
		\end{tabular}
	\end{flushleft}
	
	\newpage
	\textbf{METHOD UNDER TEST: addJob} \newline
	\textbf{Test Cases For: addJob Inputs} 
	\begin{flushleft}
		\begin{tabular}{ | l | l | l | l | l | l | l | }
			\hline
			Test Cases & startTime & endTime & Day & Salary & Requirements & Instructor \\ \hline
			Test Case no.14 & null & null & null & null & null & null \\ \hline
			Test Case no.15 & 10:00 & 15:00 & Monday & 15.00 & Bachelors & Default Instructor \\ \hline
		\end{tabular}
	\end{flushleft}
	
	\begin{flushleft}
		\begin{tabular}{ | l | l | }
			\hline
			Test Cases & Expected Output \\ \hline
			Test Case no.14 & Invalid Input for Job! \\ \hline
			Test Case no.15 & (Job gets saved in persistence layer) \\ \hline
		\end{tabular}
	\end{flushleft}

	\textbf{METHOD UNDER TEST: addJobAt} \newline
	\textbf{Test Cases For: addJobAt Inputs}
	\begin{flushleft}
		\begin{tabular}{ | l | l | l | l | l | }
			\hline
			Test Cases & Job1 Input & Index1 Input & Job2 Input & Index2 Input \\ \hline
			Test Case no.16 & Job1 & null & Job2 & null \\ \hline
			Test Case no.17 & Job1 & 1 & Job2 & 1 \\ \hline
		\end{tabular}
	\end{flushleft}
	
	\begin{flushleft}
		\begin{tabular}{ | l | l | }
			\hline
			Test Cases & Expected Output \\ \hline
			Test Case no.16 & Cannot add Job at null index! \\ \hline
			Test Case no.17 & Cannot add Job at occupied index! \\ \hline
		\end{tabular}
	\end{flushleft}

	\textbf{METHOD UNDER TEST: addOrMoveJobAt} \\
	\textbf{Test Cases For: addOrMoveJobAt Inputs}
	\begin{flushleft}
		\begin{tabular}{ | l | l | l | l | }
			\hline
			Test Cases & Job Input & Index Input & New Index Input \\ \hline
			Test Case no.18 & Job & 1 & 2 \\ \hline
			Test Case no.19 & Job & 1 & 0 \\ \hline
		\end{tabular}
	\end{flushleft}
	
	\begin{flushleft}
		\begin{tabular}{ | l | l | }
			\hline
			Test Cases & Expected Output \\ \hline
			Test Case no.18 & (Job gets moved to a new index successfully) \\ \hline
			Test Case no.19 & (Job gets moved to a new index successfully) \\ \hline
		\end{tabular}
	\end{flushleft}
\newpage
\section{Job Unit Test}
	\textbf{*Note:} Since the Classes Tutorial and Laboratory inherit from the Job class; the test cases for these two classes will be identical to the Job Class. The inputs aCourse and aInstructor are simply valid Course and Instructor inputs. \\
	\textbf{METHOD UNDER TEST: Constructor} \newline
	\textbf{Test Cases For: startTime and endTime Inputs}
	\begin{flushleft}
		\begin{tabular}{ | l | l | l | l | l | l | l | l | }
			\hline
			Test Cases & startTime & endTime & Day & Salary & Requirements & Course & Instructor \\ \hline
			Test Case no.1 & null & null & Monday & 10.00 & Bachelors & aCourse & aInstructor \\ \hline
			Test Case no.2 & "" & "" & Monday & 10.00 & Bachelors & aCourse & aInstructor \\ \hline
			Test Case no.3 & 10:00 & 9:00 & Monday & 10.00 & Bachelors & aCourse & aInstructor \\ \hline
			Test Case no.4 & 21:00 & 22:00 & Monday & 10.00 & Bachelors & aCourse & aInstructor \\ \hline
		\end{tabular}
	\end{flushleft}
	
	\begin{flushleft}
		\begin{tabular}{ | l | l | }
			\hline
			Test Cases & Expected Output \\ \hline
			Test Case no.1 & Input Time cannot be empty! \\ \hline
			Test Case no.2 & Input Time cannot be empty! \\ \hline
			Test Case no.3 & startTime cannot be before endTime! \\ \hline
			Test Case no.4 & Times cannot be outside working hours! \\ \hline
		\end{tabular}
	\end{flushleft}

	\textbf{Test Cases For: Requirements Inputs}
	\begin{flushleft}
		\begin{tabular}{ | l | l | l | l | l | l | l | l | }
			\hline
			Test Cases & startTime & endTime & Day & Salary & Requirements & Course & Instructor \\ \hline
			Test Case no.5 & 10:00 & 12:00 & Monday & 10.00 & null & aCourse & aInstructor \\ \hline
			Test Case no.6 & 10:00 & 12:00 & Monday & 10.00 & "" & aCourse & aInstructor \\ \hline
			Test Case no.7 & 10:00 & 12:00 & Monday & 10.00 & " " & aCoursee & aInstructor \\ \hline
		\end{tabular}
	\end{flushleft}
	
	\begin{flushleft}
		\begin{tabular}{ | l | l | }
			\hline
			Test Cases & Expected Output \\ \hline
			Test Case no.5 & Requirements cannot be empty! \\ \hline
			Test Case no.6 & Requirements cannot be empty! \\ \hline
			Test Case no.7 & Requirements cannot be empty! \\ \hline
		\end{tabular}
	\end{flushleft}

	\textbf{Test Cases For: Course and Instructor Inputs}
	\begin{flushleft}
		\begin{tabular}{ | l | l | l | l | l | l | l | l | }
			\hline
			Test Cases & startTime & endTime & Day & Salary & Requirements & Course & Instructor \\ \hline
			Test Case no.8 & 10:00 & 12:00 & Monday & 10.00 & PHD Student & null & aInstructor \\ \hline
			Test Case no.9 & 10:00 & 12:00 & Monday & 10.00 & PHD Student & aCourse & null \\ \hline
		\end{tabular}
	\end{flushleft}
	
	\begin{flushleft}
		\begin{tabular}{ | l | l | }
			\hline
			Test Cases & Expected Output \\ \hline
			Test Case no.8 & Invalid Course Input! \\ \hline
			Test Case no.9 & Invalid Instructor Input! \\ \hline
		\end{tabular}
	\end{flushleft}

	\newpage
	\textbf{Test Cases For: Salary Inputs}
	\begin{flushleft}
		\begin{tabular}{ | l | l | l | l | l | l | l | l | }
			\hline
			Test Cases & startTime & endTime & Day & Salary & Requirements & Course & Instructor \\ \hline
			Test Case no.10 & 10:00 & 12:00 & Monday & null & PHD Student & aCourse & aInstructor \\ \hline
			Test Case no.11 & 10:00 & 12:00 & Monday & " " & PHD Student & aCourse & aInstructor \\ \hline
			Test Case no.12 & 10:00 & 12:00 & Monday & "ten10" & PHD Student & aCourse & aInstructor \\ \hline
			Test Case no.13 & 10:00 & 12:00 & Monday & -10 & PHD Student & aCourse & aInstructor \\ \hline
		\end{tabular}
	\end{flushleft}
	
	\begin{flushleft}
		\begin{tabular}{ | l | l | }
			\hline
			Test Cases & Expected Output \\ \hline
			Test Case no.10 & Salary cannot be empty! \\ \hline
			Test Case no.11 & Salary cannot be empty! \\ \hline
			Test Case no.12 & Salary cannot have alphabetical characters! \\ \hline
			Test Case no.13 & Salary cannot be negative! \\ \hline
		\end{tabular}
	\end{flushleft}

	\textbf{Test Cases For: Day Inputs}
	\begin{flushleft}
		\begin{tabular}{ | l | l | l | l | l | l | l | l | }
			\hline
			Test Cases & startTime & endTime & Day & Salary & Requirements & Course & Instructor \\ \hline
			Test Case no.14 & 10:00 & 12:00 & null & 10.00 & PHD Student & aCourse & aInstructor \\ \hline
			Test Case no.15 & 10:00 & 12:00 & Saturday & 10.00 & PHD Student & aCourse & aInstructor \\ \hline
		\end{tabular}
	\end{flushleft}
	
	\begin{flushleft}
		\begin{tabular}{ | l | l | }
			\hline
			Test Cases & Expected Output \\ \hline
			Test Case no.14 & Date cannot be empty! \\ \hline
			Test Case no.15 & Date cannot be a weekend day! \\ \hline
		\end{tabular}
	\end{flushleft}

\newpage
\section{Profile}

\part{Integration Tests}
The integration tests are split into two parts. One part is the Java which tests the functionality of the controllers that operate in the Desktop and Mobile application in order to test the integration of multiple entity classes. The other is the PHP which tests how numerous classes can be integrated together. 

\section{Java}
All together, the Java code has tested that all classes function harmoniously as they are needed to within our Mobile and Desktop application.

\subsection{Profile Controller}
This controller tests the integration of the profile class with the courses class as well as persistence and profile manager class. This test achieved 100\% code coverage using ECLEmma. \\\\
It tested for null and empty input in all fields within the controller and ensured that it was handled. It also ensured that profiles that had incorrect input were not added to the system (were not persisted in the XML)

\subsection{Course Controller}
This controller tests the integration of the course class with the course controller, course manager and persistence. This test achieved 99.2\% code coverage. \\\\
It tested for null and empty input in all String fields in the controller and ensured that it was handled. It also checked to see that CDNs, and time budgets could not be negative. Most importantly, the integration test ensured that exceptional input did not propagate into the persistence layer.

\subsection{Application Controller}
This controller tests the integration of the application class with the application controller, application manager, profile manager, profile controller, persistence, course class, profile class, and job class. This test achieved 100\% code coverage \\\\
It tested that null and empty input in all String fields in the controller as well as negative salaries were handled by exceptions. It also tested that these errors did not propagate into persistence. 

\section{PHP}
\subsection{Job Integration}
	Given that the web client is inteded to be used by an instructor, the integration testing relates the job class to all the other classes in various scenarios and use-cases by utilising their controllers. While the unit test for profile and course used some implementation of their respective controllers, here we try to use only the controllers as much as possible for all interactions with the job class and the application controller, which handles the operations needed for the job class to be created and used. 
\end{document}