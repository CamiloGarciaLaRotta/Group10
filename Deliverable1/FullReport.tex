\documentclass[12pt]{report}

\usepackage{amsmath,amssymb,amsfonts} %In case we need math stuff
\usepackage{graphicx} %For inserting images and stuff
\usepackage{listings} %For inserting code snippets
\usepackage{enumerate} %For fancy enumeration
\usepackage[margin=2cm]{geometry} %Nice margin setting

\title{ECSE 321 - Intro to Software Engineering\\Deliverable 1}
\author{Harley Wiltzer\\Camilo Garcia La Rotta\\Jake Shnaidman\\Robert Attard\\Matthew Lesko}
\date{February 12, 2017}

\begin{document}
\pagenumbering{gobble} %No page number on title page
\maketitle
\newpage
\pagenumbering{arabic} %Arabic numeral page numbers on regular pages
\tableofcontents
\chapter{Requirements Document}
\section{Functional Requirements}
\textbf{Please Note:}
\begin{itemize}
    \item The requirement's ID is its list number, i.e. 1.1.1.
    \item The priority of requirements starts with the General Requirements from 1.1.1 to 1.1.6, followed by the Non-functional Requirements. 
    More specifically:
        \begin{enumerate}
            \item Total completion of Desktop app requirements in numerical order.
            \item Total completion of Web app in requirements numerical order.
            \item Total completion of Mobile app in requirements numerical order.
            \item Implementation of XML persistence across all platforms.
            \item Implementation of database persistence across all platforms.
            \item Intercommunication between all platforms under a centralized persistence system.
            \item Total completion of Non-functional Requirements in requirements numerical order.
        \end{enumerate} 
\end{itemize}
\subsection{General Requirements}
\begin{enumerate}[\thesubsection .1]
	\item The Teaching Assistant Management System shall include a desktop application.
	\item The Teaching Assistant Management System shall include a web application.
	\item The Teaching Assistant Management System include a mobile application.
	\item All applications (desktop, web, and mobile) shall have an XML persistence layer.
	\item All applications (desktop, web, and mobile) shall have database persistence.
	\item The three applications (desktop, web, and mobile) may be capable of communicating with
		one another.
\end{enumerate}

\subsection{Desktop Application Requirements}
\begin{enumerate}[\thesubsection .1]
	\item The desktop application shall only allow be accessible to users with administrator credentials.
	\item The desktop application shall be written in Java with the Java Swing library.
	\item The desktop application shall be capable of storing in its persistence layer a list of courses with their hours and credits.
	\item The desktop application shall be capable of manipulating a list of courses with their hours and credits from its persistence layer.
	\item The desktop application shall provide the ability to store a list of courses and their attributes listed in (1.2.3).
	\item The desktop application shall be capable of storing the student enrollment data in its persistence layer.
    \item  The desktop application shall be capable of storing the TA/Grader salaries of all McGill Departments from a CSV file onto its persistence layer.
	\item The application shall be capable of accessing TA/Grader schedules from its persistence layer.
	\item The scheduling algorithm shall never appoint work hours for prospective students that are beyond the students' available hours.
	\item The scheduling algorithm shall hire a certain TA for as many time-slots as possible for a given class with multiple lab or tutorial sessions.
	\item The scheduling algorithm shall limit individual TA/Grader hours to within the range of  a minimum of 45 for each course to 180 hours total per semester for all courses.
	\item The scheduling algorithm shall prefer graduate students to undergraduate students.
	\item The scheduling algorithm shall account for students' indicated priorities when assigning job placements.
	\item The desktop application shall provide administrators with the opportunity to review instructor modifications to the TA/Grader hours.
	\item The desktop application shall provide administrators with the opportunity to accept or reject the instructors' modifications.
	\item The desktop application shall be capable of sending job offers to the selected TA's and Graders \textit{once the administrator has explicitly accepted the placements}.
	\item The desktop application shall allow the user to perform the instructor actions described in section 1.2.
	\item The desktop application shall allow the user to perform the TA/Grader actions described in section 1.3.
\end{enumerate}

\subsection{Web Application Requirements}
\begin{enumerate}[\thesubsection .1]
	\item The web application shall only allow be accessible to users with instructor credentials.
	\item The web application shall be programmed in PHP with the use of HTML and CSS.
	\item The web application shall be capable of retrieving the course data from its persistence layer.
	\item The web application shall be capable of displaying the course data from its persistence layer.
	\item The web application shall be capable of retrieving the student enrollment data from its persistence layer.
	\item The web application shall be capable of displaying the student enrollment data from its persistence layer.
	\item The web application shall be capable of creating job postings with the attributes requisite skills and previous experience.
	\item The web application shall be able to save job postings in the persistence layer.
	\item The web application shall be capable of retrieving the course data from its persistence layer.
	\item The application shall be capable of displaying the list of TA/Grader placements from its persistence layer.
	\item The application shall allow the modification of TA/Grader placements without allowing the modifications to cause budget issues.
\end{enumerate}

\subsection{Mobile Application Requirements}
\begin{enumerate}[\thesubsection .1]
	\item The mobile application shall only allow be accessible to users with student credentials.
	\item The mobile application shall be programmed in Java for the Android operating system.
	\item The mobile application shall be created using only the tools provided in the Android UI library in Android Studio.
	\item The mobile application shall be able to create a profile that contains the users' student ID.
	\item The mobile application shall be capable of retrieving a list of job postings from its persistence layer.
	\item The mobile application shall be capable of displaying a list of job postings from its persistence layer.
	\item The mobile application shall limit the amount of applications of the user to a maximum of three.
	\item The mobile application shall allow the arbitrary ranking of job applications by the user.
	\item The mobile application shall be capable of retrieving a list of job offers from its persistence layer.
	\item The mobile application shall be capable of displaying a list of job offers from its persistence layer.
	\item The mobile application shall be capable of submitting acceptance or denial of the aforementioned job offers.
\end{enumerate}

\section{Non-functional Requirements}
\subsection{Performance Requirements}
\begin{enumerate}[\thesubsection .1]
	\item The three applications provided with the product (desktop, web, mobile) shall limit RAM usage to within 750MB.
	\item The scheduling algorithm of the desktop application shall not take longer than one minute to complete.
	\item The three applications shall provide error messages to handle unexpected behavior.
	\item The three applications shall, in case of a system crash, restart the application with the last saved persistence file.
\end{enumerate}

\subsection{Security Requirements}
\begin{enumerate}[\thesubsection .1]
	\item The desktop application shall include a secure authentication procedure to ensure that only administrators gain access.
	\item The web application shall include a secure authentication procedure to ensure that only instructors gain access.
	\item The web application shall be able to identify the current user and associate users with their modification histories.
	\item The mobile application shall include a secure authentication procedure to identify which student is currently using the software. 
	\item The mobile application shall be capable of associating student security information with the respective student account information.
	\item The persistence of administrator, instructor, and student passwords shall be achieved cryptographically, using RSA/NTRU cryptosystems.
\end{enumerate}

\subsection{Compatibility Requirements}
\begin{enumerate}[\thesubsection .1]
	\item The desktop application shall work on Windows 10, Macintosh OS X 10.5 Leopard., GNU/Linux 4.9.8, and BSD 10.3 systems.
	\item The web application shall be compatible with Google Chrome version 25, Mozilla Firefox version 50.0.0, Safari 10, Microsoft Edge, and Internet Explorer 9.
	\item The mobile application shall be compatible with Android phones running Android 4.0 (Ice Cream Sandwich) or a more recent version.
\end{enumerate}
\subsection{Graphical Requirements}
\begin{enumerate}[\thesubsection .1]
	\item The three applications provided with the product (desktop, web, mobile) shall share a common logo (app icon, desktop shortcut, web logo).
	\item The three applications shall have consistent (i.e. the same) color palettes.
	\item The three applications shall have the ability to select between Light and Dark themes for optimal comfort in a variety of lighting conditions.
	\item The color palettes (light theme and dark theme) shall be designed in a color-blind-friendly manner, to promote a common experience to all users. %for Harley's and stuff%
\end{enumerate}
\chapter{TAMAS Domain Model}
The following UMPLE code was written for the generation of the TAMAS domain model:
%\begin{figure}
\lstinputlisting[language=Java,frame=single]{model/TAMAS.ump}
%\end{figure}
\end{document}
