\documentclass[12pt]{article}
\usepackage{enumerate}
\usepackage[margin=2cm]{geometry}
\usepackage{graphicx}
\title{Publishing Use Case \\ Workflow Descriptions}

\date{}

\begin{document}
	\maketitle
		This document details the main workflow and alternative workflows of the publishing use-cases for the desktop and web applications.
	\paragraph {Desktop application}
		\subparagraph{Main Workflow\\}
			The main workflow the for publishing of a job posting from the desktop application is as follows:
			\begin{enumerate}
				\item An empty job posting is created by the admin, details can now be added to the posting.
				\item The admin can add details such as job description, applicant qualifications, salary and work hours involved as the job posting is modified,
				\item After reviewing the posting, the admin accepts that the posting is ready to be published.
				\item The admin publishes the posting, at which point it can be applied to by a student.
			\end{enumerate}
		\subparagraph{Alternative Workflows\\}
			Alternative workflows can arise for several reasons:
			\begin{itemize}
				\item If the posting is deleted at any point within the main workflow, for a posting to eventually be published the main workflow must restart, otherwise no posting will be published.
				\item If the admin does not accept the posting, the the unpublished posting must again be modified and then reassessed for acceptance.
			\end{itemize}

	\paragraph{Web application}
		\subparagraph{Main Workflow\\}
			The main workflow for the web application is almost the same as for the desktop publishing main workflow except for the following:
			\begin{itemize}
				\item The instructor is now also able to create, modify, publish and delete the job posting.
				\item While the instructor can publish the posting after it is accepted, only the admin has permission to accept a posting to be published. Only then can a posting be published by either the admin or the instructor. 
			\end{itemize}
				
		\subparagraph{Alternative Workflow\\}
			The alternative workflows for the web application are very much the same as for the desktop application except that the create, modify, delete and publish operations can now be done by the instructor.

			
\end{document}
