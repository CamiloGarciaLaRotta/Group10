\documentclass[12pt]{article}

\usepackage{amsmath,amssymb,amsfonts} %In case we need math stuff
\usepackage{graphicx} %For inserting images and stuff
\usepackage{listings} %For inserting code snippets
\usepackage{enumerate} %For fancy enumeration
\usepackage[margin=2cm]{geometry} %Nice margin setting
\graphicspath{ {C:/Users/Matthew/Downloads} }

\title{ECSE 321 - Intro to Software Engineering\\Software Architecture}
\author{Harley Wiltzer\\Camilo Garcia La Rotta\\Jake Shnaidman\\Robert Attard\\Matthew Lesko}
\date{February 19, 2017}

\begin{document}
\pagenumbering{gobble} %No page number on title page
\maketitle
\newpage
\pagenumbering{arabic} %Arabic numeral page numbers on regular pages
\tableofcontents
\section{Description}
	The software architecture comprises of two different patterns: a Model/View/Controller pattern and a Layered Architecture pattern. An "Authentication and Authorization" layer is on top of the MVC layer. Once the user is authenticated and authorized, they have access to the MVC layer. The MVC system contains three components which interact with each other: 
	\begin{itemize}
		\item View Controller
		\item View
		\item Model
	\end{itemize}
	The Model component manages the system data and associated operations on that data; it encapsulates the Job Manager, Profile Manager, and Persistence Layer. The View component defines and manages how the data is presented to the user. The View Controller component manages user interaction (key presses, mouse clicks, etc.) and passes these interactions to the View and the Model.
\section{Rationale}
	The MVC pattern was chosen because this allows the components to be changed independently. Foe example, adding a new view or changing an existing view can be done without any changes to the underlying data in the model. It allows the data to change independently of its representation and vice versa. Supports presentation of the same data in different ways with changes made in one representation shown in all of them.
	The Layered Architecture pattern was used because the user would need to first authenticate him/herself and then receive authorization in order to interact with the sublayer.
\section{Block Diagram}

\end{document}
